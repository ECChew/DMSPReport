\chapter*{Introduction}
\qquad \underline{By} : Julien\\

 Ever since the launch of Sputnik 1, the first artificial satellite in October 1957, the number of satellites launched has sharply risen and as of 2020, thousands of satellites are orbiting around the Earth. However, each of every one of those spacecrafts will eventually see their mission being stopped, usually due to a lack of resources from the satellite itself as it reaches the end of its life. \\

Those who are at the end of their lives will turn into a simple uncontrolled object that keeps orbiting around the Earth and should be avoided as another functioning satellite could take their slot as some of the most important orbits around our planet are starting to get overcrowded and the demand keeps rising.\\

Moreover, such uncontrolled objects in space can become dangerous as collisions could potentially happen and at such high velocity, those can heavily damage other spacecrafts and create even more debris.\\

As public awareness grows towards the space debris problem, our mission, Green Debris Remover (GREDER), is looking to contribute to a solution to this problem in a particular orbit, the geo stationary orbit, which is particularly overcrowded due to the many different kinds of satellite operating there. \\

This mission has been started with the following context :\\
\begin{center}
	Design of a LEO-GEO transfer vehicle for removal of suspended satellites
\end{center}
The transfer vehicle is aimed to transport a $3500$ kg satellite from GEO to atmospheric re-entry. It  will  be  stationed and  refueled in  LEO orbit ($400$ km altitude $55^\circ$ inclination). A single-stage design using  a  bi-propellant  propulsion  system has  to  be implemented. The  design  and modeling  task  shall include the  preliminary  design  and  analysis  of  the vehicle and  its propulsion system.