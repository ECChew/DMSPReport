\chapter{Definition}
\section{Mission planning}

\section{Vehicle requirements}
\begin{itemize}
    \item TL-10 The engine shall be the main propulsion system of a GEO satellite recovery vehicle.

    \item TL-11 The vehicle shall be refuelable between missions.

    \item TL-12 The vehicle shall be able to perform aerobreak maneauvers in earth's atmosphere.

    \item TL-13 The vehicle shall be able to manipulate it's flight path in earth's atmosphere using non-propulsive flight control systems.

    \item TL-14 The vehicle shall remain within the ARIANE 5 payload launch capabilities to LEO.

    \item TL-15 The vehicle shall be able to withstand debris impact of objects under 1cm of diameter with a maximum relative speed of 15km/s

    \item TL-16 The vehicle shall be able to remain on it's guided trajectory with less than 0.1 $\%$ deviation.
\end{itemize}{}


\section{Operationnal requirements}
%Reliability oriented then cost, performance is a bit less important as it's okay if the mission takes more time
\begin{itemize}
    \item TL-1 The engine shall be able to provide sufficient thrust for completion of the mission profile including a safety margin.

    \item TL-2 The engine shall be re-ignitable at least 1000 times.

    \item TL-3 The engine shall have a service life time of at least 100 missions or 25 years in orbit.

    \item TL-4 The engine's ignition and functional reliability shall be higher than 99,5$\%$.
\end{itemize}

\subsection{Environment}
\begin{itemize}
    \item TL-5 The engine shall be able to withstand the launch phase.

    \item TL-6 The engine shall be able to operate in vacuum.

    \item TL-7 The engine shall be able to operate in an ambient temperature range of 1K to 5K.

    \item TL-8 The engine shall be able to withstand the temperature gradients resulting from areas turned towards or away from the sun.

    \item TL-9 The engine shall be able to sustain space-related radiation throughout it's complete life time.
\end{itemize}


Overall, our mission will take our spacecraft between the higher layers of the Earth's atmosphere to the geostationary orbit. Thus, we will often be in a void environment which has particular effects that we need to take into account.
\subsubsection{Thermal effects}
Among those effects is the thermal problem. In space, there are only two ways of exchanging heat :
\begin{itemize}
	\item Conduction
	\item Radiation 
\end{itemize}
This has an impact on our thermal control system as we need to keep our temperature stable with a low ambient temperature but a temperature close to Earth's surface inside.
