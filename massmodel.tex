\chapter{Mass model and burning times}
\section{Mass Budget - First Iteration}

\qquad Before actually going into our mass budget, we wanted to get a reference
idea for the propellant mass so that we would be sure to be able to
achieve our \(\Delta v\). In order to get this, we decided to find a
relation between the usable propellant mass and the mass of the rest as
a ratio. This is then fixed and will also allow us to know roughly how
much propellant we need depending on the dry mass. Let \(m_{UP}\) be the
mass of usable propellant. Moreover, we would be aiming for a total initial mass of roughly $20$ to $25t$ on our last iteration. This first iteration was done with another magnet design, presented in November which consisted in two large discs of $600\ kg$ each and have then been abandoned for the second iteration.


\subsection{Coefficients \& Masses after steps}

Considering that \(ISP = 295s\) and annotating
\(\frac{m_{UP_i}}{m_{total_i}} = K_i\) with \(i\) the burn number :

\begin{longtable}[]{@{}cccc@{}}
\toprule
Step & Required \(\Delta v\) in \(m/s\) & \(K_i\) & Mass after
step\tabularnewline
\midrule
\endhead
1 & 2802.4 & 0.620 & 0.38 \(m_{initial}\)\tabularnewline
2 & 1342.2 & 0.371 & 0.239\(m_{initial}\)\tabularnewline
3 & 522.9 & 0.165 & 0.200\(m_{initial}\)\tabularnewline
Satellite caught & NA & NA & 0.2\(m_{initial}\) + 3500\tabularnewline
4 & 1487.8 & 0.402 & 0.1196\(m_{initial}\) + 2093\tabularnewline
Satellite release & NA & NA & 0.1196\(m_{initial}\) -
1407\tabularnewline
5 & 5.3 & 0.002 & 0.1194\(m_{initial}\) - 1404.186\tabularnewline
6 & 72.4 & 0.0247 &\tabularnewline
\bottomrule
\end{longtable}


\subsection{Global equation between $m_{UP}$ and
	$m_{initial}$}

\begin{longtable}[]{@{}ccc@{}}
\toprule
Step & \(\frac{m_{UP}}{m_{initial}}\) & Bias due to
debris\tabularnewline
\midrule
\endhead
1 & 0.620 & 0\tabularnewline
2 & 0.141 & 0\tabularnewline
3 & 0.039 & 0\tabularnewline
4 & 0.0804 & 1407\tabularnewline
5 & 0.00024 & -2.814\tabularnewline
6 & 0.00295 & -36.68\tabularnewline
TOTAL & 0.88359 & +1369.506\tabularnewline
\bottomrule
\end{longtable}

We then get our general relation between the usable propellant mass and
the initial mass

\[m_{prop} = 0.88359 m_{init} + 1369.506\]

And as \(m_{initial} = m_{UP} + m_{rest}\) :

\[m_{prop} = \frac 1{0.11641}\bigg[0.88359 m_{rest} + 1369.506\bigg]\]

\(m_{rest}\) includes the dry mass and the propellant required for the
ACS.

\subsection{First iteration of mass budget}


\subsubsection{Sub systems}

\begin{longtable}[]{@{}cc@{}}
\toprule
Contributor & Mass in kg\tabularnewline
\midrule
\endhead
\underline{\textbf{EPS}} & -\tabularnewline
Fuel cells & 165.6727\tabularnewline
H2 for fuel cell (tank included) & 10\tabularnewline
Cables & 20\tabularnewline
GNC & 5\tabularnewline
Batteries & 61.3333\tabularnewline
Actuators (for flaps) & 10\tabularnewline
Servos & 1\tabularnewline
\underline{\textbf{On board computer}} & 5\tabularnewline
\underline{\textbf{Telecommunications}} & 10\tabularnewline
\underline{\textbf{Thermal control}} & 10\tabularnewline
\underline{\textbf{ACS/RCS}} & -\tabularnewline
Reaction wheels & 106\tabularnewline
ACS (without propellant) & 36.16\tabularnewline
\underline{\textbf{\emph{Total}}} & 440.166\tabularnewline
\bottomrule
\end{longtable}


\subsubsection{Payload}

\begin{longtable}[]{@{}cc@{}}
\toprule
Contributor & Mass in kg\tabularnewline
\midrule
\endhead
Magnet & 1200\tabularnewline
\bottomrule
\end{longtable}


\subsubsection{Structure}

\begin{longtable}[]{@{}cc@{}}
\toprule
Contributor & Mass in kg\tabularnewline
\midrule
\endhead
Hull & 509\tabularnewline
Wing & 54\tabularnewline
Engine & 60\tabularnewline
Engine frame & 51\tabularnewline
Connectors & 25\tabularnewline
Tanks & 350\tabularnewline
Heat shield & 472\tabularnewline
\underline{\textbf{\emph{Total}}} & 1521\tabularnewline
\bottomrule
\end{longtable}


\subsubsection{Others}

\begin{longtable}[]{@{}cc@{}}
\toprule
Contributor & Mass in kg\tabularnewline
\midrule
\endhead
Catalyzer & 10\tabularnewline
Lines & 25\tabularnewline
ACS including Propellant & 672\tabularnewline
Non usable propellant (Residuals, transient, etc.) & 200\tabularnewline
Helium (including tank) & 30\tabularnewline
\underline{\textbf{\emph{Total}}} & 937\tabularnewline
\bottomrule
\end{longtable}

We then get

\[m_{rest} = m_{Sub systems} + m_{Payload} + m_{Structure} + m_{Others} = 4098.166kg\]

Which, with the previously obtained equation :

\[m_{UP} = 42\ 870.926kg\\\]

As the mixture ratio is $MR = 7.07$ and $m_{UP} = m_{UF} + m_{UOP}$\\
\begin{align*}
m_{UsableFuel} &= \frac{m_{UP}}{1 + MR} = 5\ 312kg\\
m_{UsableOxidizer} &= MR \times m_{UsableFuel} =37\ 559kg
\end{align*}

\subsubsection{Results}
We can sum this first iteration up with the following table :

	\begin{table}[H]
\centering
	
\begin{tabular}[H]{|c|c|}
	\hline
	\cellcolor{gray!50}\textbf{Contributor} & \cellcolor{green!20}\textbf{Mass} (kg)\\
	\hline
	\textbf{Structure} & $1\ 521$\\
	\hline
	\textbf{Magnets} & $1\ 200$\\
	\hline
	\textbf{Sub Systems} & $440.166$\\
	\hline
	\textbf{Tank Pressurization} & $30$\\
	\hline
	\textbf{Engine} & $60$\\
	\hline
	\textbf{Catalyzer} & $10$\\
	\hline
	\textbf{Lines} & $25$\\
	\hline
	\cellcolor{gray!50}\textbf{Dry mass} & \cellcolor{green!20} $3\ 286.166$\\
	\hline
	\textbf{Non usable propellant} & $200$\\
	\hline
	\textbf{ACS/RCS Propellant} & $142. 12$\\
	\hline
	\textbf{Usable propellant} & $42\ 870.926$\\
	\cellcolor{red!50}\textbf{Total initial mass} & \cellcolor{red!50}$46\ 969.092$\\
	\hline 
\end{tabular}
\caption{Initial mass budget}
\end{table}

This first initial mass is way over what we are targeting and there are many parameters to be refined during the next iteration.
\newpage
\section{Mass Budget - Second iteration}
\qquad After refining multiple parameters and fixing others to get more accurate values, we went into the second iteration of our mass budget. Having our $I_{SP}$ changed also required another iteration in our calculation formula between the usable propellant mass the the rest of the mass.

\subsection{Coefficients \& Masses after steps}

Considering that \(ISP = 315s\) and annotating
\(\frac{m_{UP_i}}{m_{total_i}} = K_i\) with \(i\) the burn number :

\begin{longtable}[]{@{}cccc@{}}
\toprule
Step & Required \(\Delta v\) in \(m/s\) & \(K_i\) & Mass after
step\tabularnewline
\midrule
\endhead
1 & 2802.4 & 0.596 & 0.404 \(m_{initial}\)\tabularnewline
2 & 1342.2 & 0.352 & 0.261792\(m_{initial}\)\tabularnewline
3 & 522.9 & 0.156 & 0.221\(m_{initial}\)\tabularnewline
Satellite caught & NA & NA & 0.221\(m_{initial}\) + 3500\tabularnewline
4 & 1487.8 & 0.382 & 0.137\(m_{initial}\) + 2163\tabularnewline
Satellite release & NA & NA & 0.137\(m_{initial}\) -1337\tabularnewline
5 & 5.3 & 0.0017 & 0.1368\(m_{initial}\) - 1334.73\tabularnewline
6 & 72.4 & 0.023 &\tabularnewline
\bottomrule
\end{longtable}


\subsection{\texorpdfstring{Global equation between \(m_{UP}\) and
		\(m_{initial}\)}{Global equation between m\_\{UP\} and m\_\{initial\}}}
\begin{longtable}[]{@{}ccc@{}}
\toprule
Step & \(\frac{m_{UP}}{m_{initial}}\) & Bias due to
debris\tabularnewline
\midrule
\endhead
1 & 0.596 & 0\tabularnewline
2 & 0.142 & 0\tabularnewline
3 & 0.041 & 0\tabularnewline
4 & 0.084 & 1337\tabularnewline
5 & 0.0002 & -2.273\tabularnewline
6 & 0.0032 & -30.699\tabularnewline
TOTAL & 0.8664 & +1304.028\tabularnewline
\bottomrule
\end{longtable}



This time our equation between those two masses is given by 
\begin{equation}
m_{UsableProp} = \frac 1{0.1336}\bigg[0.8664 m_{rest} + 1304.028\bigg]
\end{equation}

\subsection{Second iteration of mass budget}

\qquad As our way of presenting our first iteration of the mass budget didn't seems clear enough to us, we decided to present it in another, more logical way :
\textbf{\underline{Structure}}
\begin{center}
\begin{tabular}[H]{|c|c|}
	\hline
	\cellcolor{gray!50}Contributor & \cellcolor{gray!50}Mass (kg)\\
	\hline
	Hull & $192$\\
	\hline
	Tanks (including non usable propellant) & $700$\\
	\hline
	Wings & $136$\\
	\hline
	Lines & $60$\\
	\hline
	Connectors & $16$\\
	\hline
	$H_2$ tank & $12$\\
	\hline
	\cellcolor{green!30}\textbf{Structure} & \textbf{$1\ 116$}\\
	\hline
\end{tabular}
\end{center}

\textbf{\underline{Electrical related contributors}}
\begin{center}
\begin{tabular}[H]{|c|c|}
	\hline
	\cellcolor{gray!50}Contributor & \cellcolor{gray!50}Mass (kg)\\
	\hline
	Batteries & $241$\\
	\hline
	Fuel cells & $202$\\
	\hline
	On Board Computer & $5$\\
	\hline
	Cables & $20$\\
	\hline
	$H_2$ for fuel cells & $5$\\
	\hline
	Wing actuators & $10$\\
	\hline
	Telecommunications & $10$\\
	\hline
	GNC & $5$\\
	\hline
	Thermal Control & $10$\\
	\hline
	Magnets (Payload) & $25.65$\\
	\hline
	\cellcolor{green!30}\textbf{Electrical related contributors} & \textbf{$533.65$}\\
	\hline
\end{tabular}
\end{center}

\textbf{\underline{ACS and RCS}}
\begin{center}
\begin{tabular}[H]{|c|c|}
	\hline
	\cellcolor{gray!50}Contributor & \cellcolor{gray!50}Mass (kg)\\
	\hline
	Thrusters & $36$\\
	\hline
	$H_2O_2$ & $90$\\
	\hline
	Reaction wheels & $106$\\
	\hline
	\cellcolor{green!30}\textbf{ACS \& RCS} & \textbf{$232$}\\
	\hline
\end{tabular}
\end{center}

\textbf{\underline{Propulsion}}
\begin{center}
\begin{tabular}[H]{|c|c|}
	\hline
	\cellcolor{gray!50}Contributor & \cellcolor{gray!50}Mass (kg)\\
	\hline
	Engine & $93$\\
	\hline
	Turbopumps & $25$\\
	\hline
	Pressurization ($He$) & $1.3$\\
	\hline
	Catalyzer & $30$\\
	\hline
	\cellcolor{green!30}\textbf{Propulsion} & \textbf{$149.3$}\\
	\hline
\end{tabular}
\end{center}

With those tables, we can deduce $m_{rest}$ :
\begin{align}
m_{rest} &= m_{Structure} + m_{Elec} + m_{ACS\&RCS} + m_{Propulsion}\\
m_{rest} &= 2\ 030.95kg
\end{align}
Thus,
\begin{align}
m_{UsableProp} &= \frac 1{0.1336}\bigg[0.8664 m_{rest} + 1304.028\bigg]\\
m_{UsableProp} &= 22\ 931kg\\
m_{Fuel} &= \frac{m_{UsableProp}}{MR+1}\\
m_{Fuel} &= 2841.6kg\\
m_{Ox} &= m_{UsableProp} - m_{Fuel}\\
m_{Ox} &= 20\ 089.89kg\\
m_0 &= 24\ 962.41kg
\end{align}



In this second iteration with a better \(I_{sp}\) and refined values for all of the contributors, we have a large improvement as our initial mass decreased drastically.
\newpage
\section{Frozen information}[H]
After our second iteration of the mass budget, we decided to make a list of the fixed values that we will work around in our further design.
\subsection{Frozen points}
\begin{itemize}
\item We will do $20$ aerobrakes
\item We will have a separate tank design
\item $H_2O_2$ will be pressurized by its decomposition
\item The decomposition control will be managed by rotation of the spacecraft
\item ACS/RCS Layout similar to the Space Shuttle
\item $H_2O_2$ catalyzers separate
\item $H_2O_2/O_2$ separation via thermodynamic properties
\item $H_2/O_2$ will be used in fuel cells to produce energy
\end{itemize}
\begin{table}
\centering
\begin{tabular}[H]{|c|c|c|}
	\hline
	\cellcolor{gray!50}Data & \cellcolor{gray!50}Value & \cellcolor{gray!50}Unit\\
	\hline
	Empty raw mass & $2\ 031$ & kg\\
	\hline
	Usable propellant & $22\ 931$ &kg\\
	\hline
	\cellcolor{green!50}Total mass & \cellcolor{green!50}$24\ 962$ & \cellcolor{green!50}kg\\
	\hline
	Flowrate & $10$ & kg/s\\
	\hline
	Rocket diameter & $2$ & $m$\\
	\hline
	$I_{sp_{vacuum}}$ & $335$ & s\\
	\hline
	Thrust $F=\dot m I_{sp} g_0$ & $32 863.5$ &N\\
	\hline
	Mixture Ratio & $7.07$ & -\\
	\hline
	Wall thickness & $TBA$ & m\\
	\hline
	$H_2O_2$ internal pressure & $1.35$ & bar\\
	\hline
\end{tabular}
\caption{Frozen information}
\end{table}

\newpage
\section{Mass Budget - Final iteration}
As the fixed $I_{sp}$ has been refined as well as other parameters, we went into our final iteration of the mass budget with the same process as the two previous ones.
\subsection{Coefficients \& Masses after steps}

Considering that \(ISP = 335s\) and annotating
\(\frac{m_{UP_i}}{m_{total_i}} = K_i\) with \(i\) the burn number :

\begin{longtable}[]{@{}cccc@{}}
\toprule
Step & Required \(\Delta v\) in \(m/s\) & \(K_i\) & Mass after
step\tabularnewline
\midrule
\endhead
1 & 2802.4 & 0.574 & 0.426 \(m_{initial}\)\tabularnewline
2 & 1342.2 & 0.335 & 0.283\(m_{initial}\)\tabularnewline
3 & 522.9 & 0.148 & 0.241\(m_{initial}\)\tabularnewline
Satellite caught & NA & NA & 0.241\(m_{initial}\) + 3500\tabularnewline
4 & 1487.8 & 0.364 & 0.153\(m_{initial}\) + 2226\tabularnewline
Satellite release & NA & NA & 0.153\(m_{initial}\) -1274\tabularnewline
5 & 5.3 & 0.0016 & 0.1528\(m_{initial}\) - 1271.96\tabularnewline
6 & 72.4 & 0.0218 & (0.1495$m_{initial}$ - 1244.23)\tabularnewline
\bottomrule
\caption{Coefficients and masses after steps}
\end{longtable}


\subsection{\texorpdfstring{Global equation between \(m_{UP}\) and
		\(m_{initial}\)}{Global equation between m\_\{UP\} and m\_\{initial\}}}



\begin{longtable}[]{@{}ccc@{}}
\toprule
Step & \(\frac{m_{UP}}{m_{initial}}\) & Bias due to
debris\tabularnewline
\midrule
\endhead
1 & 0.574 & 0\tabularnewline
2 & 0.143 & 0\tabularnewline
3 & 0.042 & 0\tabularnewline
4 & 0.088 & 1274\tabularnewline
5 & 0.0002 & -2.038\tabularnewline
6 & 0.0033 & -27.73\tabularnewline
TOTAL & 0.8505 & +1244.232\tabularnewline
\bottomrule
\end{longtable}
Thus,
$$
m_{UsablePropellant} = \frac 1{0.1495}[0.8505m_{rest}+1244.232]
$$
\subsection{Detailed contributors}
\subsubsection{Structure}
\begin{table}

\begin{center}
\begin{tabular}[H]{|c|c|}
	\hline
	\cellcolor{gray!50}Contributor & \cellcolor{gray!50}Mass (kg)\\
	\hline
	Hull & $192$\\
	\hline
	Tanks & $700$\\
	\hline
	Wings & $136$\\
	\hline
	Lines & $70$\\
	\hline
	Connectors & $16$\\
	\hline
	Brackets & $35$\\
	\hline
	$H_2$ Tanks & $12$\\
	\hline
	\cellcolor{green!30}\textbf{Structure} & \textbf{$1161$}\\
	\hline
\end{tabular}
\end{center}
\label{Mass budget - Structure}	
\end{table}
\subsubsection{Electrical systems}
\begin{table}
	
\begin{center}
\begin{tabular}[H]{|c|c|}
	\hline
	\cellcolor{gray!50}Contributor & \cellcolor{gray!50}Mass (kg)\\
	\hline
	Batteries & $388.66$\\
	\hline
	Fuel cell & $202$\\
	\hline
	OBC & $10$\\
	\hline
	Cables & $57$\\
	\hline
	$H_2$ for fuel cells & $5$\\
	\hline
	Wing actuators & $10$\\
	\hline
	Data transmission & $10$\\
	\hline
	GNC & $10$\\
	\hline
	Thermal control & $20$\\
	\hline
	Magnets & $25.65$\\
	\hline
	\cellcolor{green!30}\textbf{Electrical systems} & \textbf{$738.31$}\\
	\hline
\end{tabular}
\end{center}
\caption{Mass budget - Electrical systems}
\end{table}
The mass of the batteries is given by $m_{Bat} = \frac{E_{pumps} + E_{magnet}}{ED_{Lithium}}$ with a considered energy density of $100Wh/kg$
\subsubsection{Attitude control}
\begin{table}
	
\begin{center}
\begin{tabular}[H]{|c|c|}
	\hline
	\cellcolor{gray!50}Contributor & \cellcolor{gray!50}Mass (kg)\\
	\hline
	Thrusters & $36$\\
	\hline
	$H_2O_2$ for the ACS & $90$\\
	\hline
	
	Reaction wheels & $106$\\
	\hline
	\cellcolor{green!30}\textbf{Attitude control} & \textbf{$232$}\\
	\hline
\end{tabular}
\end{center}
\caption{Mass Budget - Attitude Control}
\end{table}
\subsubsection{Propulsion}
\begin{table}
	
\begin{center}
\begin{tabular}[H]{|c|c|}
	\hline
	\cellcolor{gray!50}Contributor & \cellcolor{gray!50}Mass (kg)\\
	\hline
	Engine & $93$\\
	\hline
	Turbopumps + Electric motors & $170$\\
	\hline
	$He$ for pressurization & $1.3$\\
	\hline
	Catalyzer & $40.86$\\
	\hline
	\cellcolor{green!30}\textbf{Propulsion} & \textbf{$305.16$}\\
	\hline
\end{tabular}
\end{center}
\caption{Mass Budget - Propulsion}

\end{table}
\subsection{Final mass budget}

\begin{table}[H]
	\centering
\begin{tabular}[H]{|c|c|}
	\hline
	\cellcolor{gray!50}Contributor & \cellcolor{gray!50}Mass (kg)\\
	\hline
	Structure & $1161$\\
	\hline
	Electrical systems & $738.31$\\
	\hline
	Attitude control & $232$\\
	\hline
	Heat shield & $360$\\
	\hline
	Propulsion & $305.16$\\
	\hline
	\cellcolor{green!30}\textbf{$m_{rest}$} & \textbf{$2796.5$}\\
	\hline
	\cellcolor{green!30}\textbf{$m_{UP}$} with a $5\%$ performance window & \textbf{$25\ 505$}\\
	\hline
	\cellcolor{green!30}\textbf{$m_{Fuel}$}  & \textbf{$3160. 5$}\\
	\hline
	\cellcolor{green!30}\textbf{$m_{Ox}$}  & \textbf{$22\ 345$}\\
	\hline
	\cellcolor{red!30}\textbf{Initial wet mass}  & \textbf{$28\ 302$}\\
	\hline
\end{tabular}
\caption{Final mass budget}
\end{table}
\section{Burn times}
\qquad From the tables of $\Delta v$ and related mass-ratios (from the Tsiolkovsky Equation), we can get the burn times as, for each step, the burn time is defined by : 
\begin{equation}
	t_{burn} = \frac{\Delta m_{step}}{\dot{m}}
\end{equation}
With $\dot{m}$ being the general mass flow of $10$ kg/s. We separate our first step into two different burns. 
\begin{equation}
	\Delta m_j  = m_{j-1}
\end{equation}
\begin{table}
	\centering
	\begin{tabular}{|c|c|}
		\hline
		Step & Mass after step (kg)\\
		\hline
		1 & 12 056.652\\
		\hline
		2 & 8 009.466\\
		\hline
		3 & 6 820.782\\
		\hline
		Catch & 9 754.742\\
		\hline
		4 & 6 556.206\\
		\hline
		Release & 3 056.206\\
		\hline
		5 & 3 052.586\\
		\hline
		6 & 2 986.039\\
		\hline
	\end{tabular}
	\caption{Masses after step} \label{tabmass}
\end{table}
\begin{table}[H]
	\centering
	\begin{tabular}{|c|c|c|c|}
		\hline
		Step & $\Delta v_{step}(m/s)$ &  $\Delta m_{step} (kg)$ & $t_{burn} (s)$\\
		\hline
		1 & 2802.4 &  16 245.35 & 1 624.535 (separate into two burns)\\
		\hline
		2 & 1342.2 & 4 047.19 & 404.719\\
		\hline
		3 & 522.9 & 1 188.684 & 118.868\\
		\hline
		4 & 1487.8 & 3 198.536 & 319.854\\
		\hline
		5 & 5.3 & 3.62 & 0.362\\
		\hline
		6 & 72.4 & 66.817 & 6.682\\
		\hline
		TOTAL & 6 233 &  & 2 475.02\\
		\hline
	\end{tabular}
\end{table}