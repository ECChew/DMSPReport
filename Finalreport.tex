% A LaTeX (non-official) template for ISAE projects reports
% Copyright (C) 2014 Damien Roque
% Version: 0.2
% Author: Damien Roque <damien.roque_AT_isae.fr>

\documentclass[a4paper,12pt,calibri,oneside,openany]{book}
\usepackage{geometry}
\usepackage[utf8]{inputenc}
\usepackage[T1]{fontenc}
%\usepackage[french]{babel} % If you write in French
\usepackage[english]{babel} % If you write in English
\usepackage{a4wide}
\usepackage{graphicx}
\graphicspath{{images/}}
\usepackage{subfig}
\usepackage{tikz}
\usetikzlibrary{shapes,arrows}
\usepackage{pgfplots}
\pgfplotsset{compat=newest}
\pgfplotsset{plot coordinates/math parser=false}
\newlength\figureheight
\newlength\figurewidth
\pgfkeys{/pgf/number format/.cd,
set decimal separator={,\!},
1000 sep={\,},
}
\usepackage{ifthen}
\usepackage{ifpdf}
\usepackage{pdfpages}
\ifpdf
\usepackage[pdftex]{hyperref}
\else
\usepackage{hyperref}
\fi
\usepackage{color}
\hypersetup{%
colorlinks=true,
linkcolor=black,
citecolor=black,
urlcolor=black}
\usepackage{float}
\renewcommand{\baselinestretch}{1.05}
\usepackage{fancyhdr}
\pagestyle{fancy}
\fancyfoot{}
\fancyhead[LE,RO]{\textbf{Page \thepage/\pageref{LastPage}}}
\fancyhead[RE]{\bfseries\nouppercase{\leftmark}}
\fancyhead[LO]{\bfseries\nouppercase{\rightmark}}
\setlength{\headheight}{15pt}

\let\headruleORIG\headrule
\renewcommand{\headrule}{\color{black} \headruleORIG}
\renewcommand{\headrulewidth}{1.0pt}
\usepackage{colortbl}
\arrayrulecolor{black}


\usepackage{lastpage}
\renewcommand\headrulewidth{1pt}
\fancyfoot[L]{DMSP}
\renewcommand\footrulewidth{1pt}
\fancyfoot[C]{GREDER Project}
\fancyfoot[R]{\today}
\makeatletter
\def\@textbottom{\vskip \z@ \@plus 1pt}
\let\@texttop\relax
\makeatother

\makeatletter
\def\cleardoublepage{\clearpage\if@twoside \ifodd\c@page\else%
  \hbox{}%
  \thispagestyle{empty}%
  \newpage%
  \if@twocolumn\hbox{}\newpage\fi\fi\fi}
\makeatother

\usepackage{amsthm}
\usepackage{amssymb,amsmath,bbm}
\usepackage{array}
\usepackage{bm}
\usepackage{multirow}
\usepackage[footnote]{acronym}
\usepackage{float}
\usepackage{wasysym}
\usepackage{wrapfig}
\usepackage{url}
\usepackage{eurosym}
\usepackage{array}
\usepackage{xcolor}
\usepackage{supertabular}
%\usepackage{geometry}
\usepackage{pdflscape}
\usepackage{calrsfs}
\usepackage{longtable, booktabs}
\usepackage{minted}
\newcommand*{\SET}[1]  {\ensuremath{\mathbf{#1}}}
\newcommand*{\VEC}[1]  {\ensuremath{\boldsymbol{#1}}}
\newcommand*{\FAM}[1]  {\ensuremath{\boldsymbol{#1}}}
\newcommand*{\MAT}[1]  {\ensuremath{\boldsymbol{#1}}}
\newcommand*{\OP}[1]  {\ensuremath{\mathrm{#1}}}
\newcommand*{\NORM}[1]  {\ensuremath{\left\|#1\right\|}}
\newcommand*{\DPR}[2]  {\ensuremath{\left \langle #1,#2 \right \rangle}}
\newcommand*{\calbf}[1]  {\ensuremath{\boldsymbol{\mathcal{#1}}}}
\newcommand*{\shift}[1]  {\ensuremath{\boldsymbol{#1}}}

\newcommand{\eqdef}{\stackrel{\mathrm{def}}{=}}
\newcommand{\argmax}{\operatornamewithlimits{argmax}}
\newcommand{\argmin}{\operatornamewithlimits{argmin}}
\newcommand{\ud}{\, \mathrm{d}}
\newcommand{\vect}{\text{Vect}}
\newcommand{\sinc}{\ensuremath{\mathrm{sinc}}}
\newcommand{\esp}{\ensuremath{\mathbb{E}}}
\newcommand{\hilbert}{\ensuremath{\mathcal{H}}}
\newcommand{\fourier}{\ensuremath{\mathcal{F}}}
\newcommand{\sgn}{\text{sgn}}
\newcommand{\intTT}{\int_{-T}^{T}}
\newcommand{\intT}{\int_{-\frac{T}{2}}^{\frac{T}{2}}}
\newcommand{\intinf}{\int_{-\infty}^{+\infty}}
\newcommand{\Sh}{\ensuremath{\boldsymbol{S}}}
\newcommand{\C}{\SET{C}}
\newcommand{\R}{\SET{R}}
\newcommand{\Z}{\SET{Z}}
\newcommand{\N}{\SET{N}}
\newcommand{\K}{\SET{K}}
\newcommand{\reel}{\mathcal{R}}
\newcommand{\imag}{\mathcal{I}}
\newcommand{\cmnr}{c_{m,n}^\reel}
\newcommand{\cmni}{c_{m,n}^\imag}
\newcommand{\cnr}{c_{n}^\reel}
\newcommand{\cni}{c_{n}^\imag}
\newcommand{\tproto}{g}
\newcommand{\rproto}{\check{g}}
\newcommand{\LR}{\mathcal{L}_2(\SET{R})}
\newcommand{\LZ}{\ell_2(\SET{Z})}
\newcommand{\LZI}[1]{\ell_2(\SET{#1})}
\newcommand{\LZZ}{\ell_2(\SET{Z}^2)}
\newcommand{\diag}{\operatorname{diag}}
\newcommand{\noise}{z}
\newcommand{\Noise}{Z}
\newcommand{\filtnoise}{\zeta}
\newcommand{\tp}{g}
\newcommand{\rp}{\check{g}}
\newcommand{\TP}{G}
\newcommand{\RP}{\check{G}}
\newcommand{\dmin}{d_{\mathrm{min}}}
\newcommand{\Dmin}{D_{\mathrm{min}}}
\newcommand{\Image}{\ensuremath{\text{Im}}}
\newcommand{\Span}{\ensuremath{\text{Span}}}

\newcommand{\anfr}[1]{{\bfseries\underline{#1}}}

\newtheoremstyle{break}
  {11pt}{11pt}%
  {\itshape}{}%
  {\bfseries}{}%
  {\newline}{}%
\theoremstyle{break}

%\theoremstyle{definition}
\newtheorem{definition}{Définition}[chapter]

%\theoremstyle{definition}
\newtheorem{theoreme}{Théorème}[chapter]

%\theoremstyle{remark}
\newtheorem{remarque}{Remarque}[chapter]

%\theoremstyle{plain}
\newtheorem{propriete}{Propriété}[chapter]
\newtheorem{exemple}{Exemple}[chapter]



%\sloppy
\usepackage{wrapfig}
\usepackage{enumitem}
\usepackage{pifont}
\usepackage{makeidx}
\usepackage{setspace}
\usepackage{xr}
\usepackage{zref}
\usepackage{zref-xr}
\usepackage{xr-hyper}
\makeindex
\usepackage[xindy]{glossaries}
\usepackage{adjustbox}
\makeglossaries
%\loadglsentries{glossaire.tex}




\begin{document}

\renewcommand{\bibname}{Bibliographie et Webographie}
%%%%%%%%%%%%%%%%%%
%%% First page %%%
%%%%%%%%%%%%%%%%%%

\begin{titlepage}
\begin{center}

\includegraphics[width=0.5\textwidth]{logohsb}\\[1cm]

%{\large Étudiants ingénieurs en aérospatial}\\[0.5cm]

%{\large DMSP}\\[0.5cm]

% Title
\rule{\linewidth}{0.5mm} \\[0.4cm]
{ \huge  \textbf{GREDER} \\[0.1cm] \textbf{G}reen \textbf{RE}-usable \textbf{DE}bris \textbf{R}emover }
\rule{\linewidth}{0.5mm} \\[1.5cm]
%\vspace{1cm}
\begin{center}
	\includegraphics[height=6cm]{greder}
\end{center}
\vspace{0.8cm}
% Author and supervisor
\noindent
\begin{minipage}{0.4\textwidth}
  \begin{flushleft} \large
    \emph{Authors :}\\
    Tim \textbf{\textit{Lewis}}\\
    Alina \textbf{\textit{Trifunovic}}\\
    Lukas \textbf{\textit{Krause}}\\
    Alexis \textbf{\textit{Rolin}}\\
    Julien \textbf{\textit{Huynh}}\
  \end{flushleft}
\end{minipage}%
\begin{minipage}{0.4\textwidth}
  \begin{flushright} \large
    \emph{Supervising professor :} \\
    Prof. Dr.-Ing. Uwe \textit{Apel}\\
  \end{flushright}
\end{minipage}

\vfill

% Bottom of the page
{\large Version 0.9.3\\ \today}

\end{center}
\end{titlepage}

%%%%%%%%%%%%%%%%%%%%%%%%%%%%%
%%% Non-significant pages %%%
%%%%%%%%%%%%%%%%%%%%%%%%%%%%%

\frontmatter

%\chapter*{Remerciements}


\tableofcontents

\mainmatter
\pagestyle{fancy}
%%%%%%%%%%%%%%%%%%%%%%%%%%%%%%%%%%%%%%%%%%%%
%%% Content of the report and references %%%
%%%%%%%%%%%%%%%%%%%%%%%%%%%%%%%%%%%%%%%%%%%%



\chapter*{Introduction}
\qquad Ever since the launch of Sputnik 1, the first artificial satellite in October 1957, the number of satellites launched has sharply risen and as of 2020, thousands of satellites are orbiting around the Earth. However, each of every one of those spacecrafts will eventually see their mission being stopped, usually due to a lack of resources from the satellite itself as it reaches the end of its life. \\

Those who are at the end of their lives will turn into a simple uncontrolled object that keeps orbiting around the Earth and should be avoided as another functioning satellite could take their slot as some of the most important orbits around our planet are starting to get overcrowded and the demand keeps rising.\\

Moreover, such uncontrolled objects in space can become dangerous as collisions could potentially happen and at such high velocity, those can heavily damage other spacecrafts and create even more debris.\\

As public awareness grows towards the space debris problem, our mission, Green Debris Remover (GREDER), is looking to contribute to a solution to this problem in a particular orbit, the geo stationary orbit, which is particularly overcrowded due to the many different kinds of satellite operating there. \\

This mission has been started with the following context :\\
\begin{center}
	Design of a LEO-GEO transfer vehicle for removal of suspended satellites
\end{center}
The transfer vehicle is aimed to transport a $3500$ kg satellite from GEO to atmospheric re-entry. It  will  be  stationed and  refueled in  LEO orbit ($400$ km altitude $55^\circ$ inclination). A single-stage design using  a  bi-propellant  propulsion  system has  to  be implemented. The  design  and modeling  task  shall include the  preliminary  design  and  analysis  of  the vehicle and  its propulsion system.
\chapter{Schedule}
\qquad \underline{Written by} : Alina\\

 At the start of the project a dedicated group meeting was performed in order to agree on a common project sequence, tasks and challenges as well as work distribution. This group meeting was deemed essential to structure the work packages and to achieve a consolidated baseline for the whole project including time line.\\

The result is a complex MS project Gantt diagram, which can be found in \nameref{sec:annex1}.\\
\section{Initial Schedule}
\qquad The first version of the schedule starts with a project Kick-Off in October which is afterwards followed by a short planning phase. In this planning phase issues as scheduling, work distribution and scope of the project were addressed.\\

Subsequently, the definition phase started. Within this phase, the vehicle requirements were defined and the mission was planned, calculated and visualized in MATLAB. The outcomes of the definition phase are the boundary requirements which are set to provide a frame for both: the vehicle itself as well as the propulsion system. The requirements were defined at the beginning of the project and were verified after project completion.\\

Upon definition of the boundary requirements, the specification phase started. Within this phase, different propellant combinations were identified, discussed and compared. Additionally a first mass budget was calculated. The result of this phase is the system specification.\\

The sequence of the project includes several presentations. The first one was performed in October for a quick overview on the project planning. The second one after the boundary requirements and system specification was set. 
After this presentation, the vehicle and sub system design phase started. This phase included major parts of the work packages including propulsion system design with all sub-assemblies as RACS/ACS, propellant tanks, feeding and pressurization system, turbo pumps, catalyzer, engine, injector and nozzle. The outcome of this phase is a preliminary vehicle design and sub system design which was presented in the mid-term presentation.\\

As a last major work package the simulation phase started. The whole system was simulated including all sub systems and additionally the complex H2O2 decomposition regulation. The final presentation was performed after all tasks were completed and the simulation was finalized.\\

An overview the compressed initial schedule is shown in \autoref{Fig1}. The detailed Gantt schedule can be found in Annex 1.

\begin{figure}[H]
	\centering\includegraphics[width=\linewidth]{initialschedule}
	\caption{Initial schedule}\label{Fig1}
\end{figure}
\clearpage
\section{Final schedule – comparison as planned and as achieved}

\qquad As usual in project management and project work, not all milestones were achieved in time. As it is shown in the compressed final schedule in \autoref{Fig2}, the finalized vehicle design, the finalized sub system design and the corresponding simulation shifted within the project schedule (\autoref{Fig2}, shown in red). Nevertheless all work packages have been successfully completed until Milestone 4, the final presentation.

\begin{figure}[H]
	\centering\includegraphics[width=\linewidth]{finalschedule}
	\caption{Final schedule - comparison as planned and as achieved}\label{Fig2}
\end{figure}


\documentclass{article}
\usepackage[utf8]{inputenc}

\title{DMSP Requirements}
\author{lukas }
\date{November 2019}

\begin{document}

\maketitle

\section{Introduction}

Operation

TL-1 The engine shall be able to provide sufficient thrust for completion of the mission profile including a safety margin.

TL-2 The engine shall be re-ignitable at least 1000 times.

TL-3 The engine shall have a service life time of at least 100 missions or 25 years in orbit.

TL-4 The engine's ignition and functional reliability shall be higher than 99,5$\%$.

Environment

TL-5 The engine shall be able to withstand the launch phase.

TL-6 The engine shall be able to operate in vacuum.

TL-7 The engine shall be able to operate in an ambient temperature range of 1K to 5K.

TL-8 The engine shall be able to withstand the temperature gradients resulting from areas turned towards or away from the sun.

TL-9 The engine shall be able to sustain space-related radiation throughout it's complete life time.

Vehicle

TL-10 The engine shall be the main propulsion system of a GEO satellite recovery vehicle.

TL-11 The vehicle shall be refuelable between missions.

TL-12 The vehicle shall be able to perform aerobreak maneauvers in earth's atmosphere.

TL-13 The vehicle shall be able to manipulate it's flight path in earth's atmosphere using non-propulsive flight control systems.

TL-14 The vehicle shall remain within the ARIANE 5 payload launch capabilities to LEO.

TL-15 The vehicle shall be able to withstand debris impact of objects under 1cm of diameter with a maximum relative speed of 15km/s

TL-16 The vehicle shall be able to remain on it's guided trajectory with less than 0.1$\%$ deviation.
\end{document}

\chapter{Launcher requirements and launch envelope}

From our initial requirements we set to stay within certain margins of size and mass. These two parameters create constraint for the launch. 
Indeed, we have to analyze the different launcher on the market and their capabilities. 
Thus, we chose to stay within the capabilities of two of the most used actual launchers: Ariane 6.4 and Falcon 9. \\

These launchers have almost the same launch capability; the Falcon 9 can send $22.8$ t in LEO and Ariane 6.4 can send $21.6$ t to LEO.
The other important characteristic is the dimension of the fairing.
Here we have two different launchers with almost identical size; the Falcon 9 can carry a payload up to $11.4$ m long and $4.6$ m wide, whereas Ariane 6.4 can carry up to $18$ m long and $4.57$ m wide but as the Ariane's fairing is very elongated the true usable size will be more around $12$ m by $4.57$ m. \\

To allow us to correctly design our spacecraft we need to create a theoretical envelope based on these two launchers. For that we set the following dimension: a diameter of $4.5$ m and a total length of $11.4$ m. These dimensions are lower than the two launchers to unsure to have a safety margin.

\clearpage

To sum up, here is the 3 fairing we spoke about:

\begin{figure}[H]
    \centering
    \includegraphics[width=\linewidth]{enveloppe}
    \caption{Falcon 9 - Ariane 6 - combined fairing}
    \label{fig:my_label}
\end{figure}

In case, during the design process, the spacecraft cannot stay behind the 22t to LEO launch capability of the Falcon 9, we plan to use the bigger Falcon Heavy which can send 60t to LEO. This launcher will be the last choice, we want at most to stay within our pre-requirement. \\

We first think to use the structure of our spacecraft, which is aerodynamic, as a fairing so we have less constraint on the size but as it has wings and tail it will generate a small lift but sufficient to destabilize the launcher and might cause a crash. So we decided to forget this idea and to stay within a classic fairing. \\

Through our project we are going to see if our requirements can meet our calculations and we will make a choice of launcher. 
\chapter{Mission analysis }
\section{Mission Summary}
The objective of the mission is the deorbiting of a GEO satellite. The first step to developing the
propulsion system is to break down the mission segments and calculate the delta-V for each of them.
Then after the $\Delta v$ and some mission constraints are clear the detailed mission profile needs to be chosen and optimized.\\

The mission is broken down into two major phases. Phase 1 is to reach the satellites orbit and match its
velocity to enable the capturing process. Phase 2 is deorbitng te satellite and retuning to the original LEO orbit.
\subsection{Phase 1}
\begin{center}
	Transferring the spacecraft from LEO at $55^\circ$ inclination to GEO at $0^\circ$ inclination
\end{center}
First concept : \\

\begin{itemize}
	\item Burn 1.1 : LEO to GTO at $55^\circ$
	\item Burn 1.2 : Inclination change to $0^\circ$ at Apogee of GTO
	\item Burn 1.3 : GTO to GEO at $0^\circ$
\end{itemize}
To calculate the required delta-V for a burn to change the velocity without changing the direction of the spacecraft is simply identifying the difference in value of the starting and target vector. So for Burn 1.1 the orbital speed of LEO at 400 km needs to be calculated as well as the perigee speed of an elliptic orbit with 400 km as perigee and GEO altitude as apogee. To calculate these values several Matlab functions were written:

\begin{minted}[fontsize=\footnotesize, linenos, autogobble, breaklines]{matlab}
function a = v(r) % velocity of a circular orbit with Radius r
a = sqrt(mu/r);
end
\end{minted}
Then the function was used for the calculation with $R_{LEO} = 6778$ km
\begin{equation}
	v_{LEO} = v(R_{LEO}) = 7669 m/s
\end{equation}
The target velocity for the perigee of the GTO is calculated with the following function:
\begin{minted}[fontsize=\footnotesize, linenos, autogobble, breaklines]{matlab}
function a = vp(r, R) % perigee velocity of elliptical orbit with perigee r and apogee R
if r<R %if clause to prevent mistaken input
a = sqrt(2*mu/r-2*mu/(r+R));
end
end
\end{minted}

With $R_{GEO} = 42164$ km, the speed is calculated:
\begin{equation}
	v_{gto_p} = v_p(R_leo, R_geo) = 10066 m/s
\end{equation}
The difference between the two values is the required $\Delta v, \Delta v_1 = 2398$ m/s.\\

To calculate the delta-V needed for an inclination change the following function was used:
\begin{minted}[fontsize=\footnotesize, linenos, autogobble, breaklines]{matlab}
function a= dVi(i, v) % delta-V required for inclination change of i (deg) and velocity v
a= 2*v*sin(deg2rad(i)/2);
end
\end{minted}

The velocity passed to the function is the velocity at which the inclination change shall be performed.
Here the apogee velocity of the GTO $v_{gto_a} = 1618$ m/s was used:
\begin{equation}
	\Delta v{inc_1} = \Delta v_i(i, v_{gto_a})
\end{equation}

Lastly, the velocity change from the apogee of the GTO $v_{gto_a}$ and the velocity at GEO $v_{geo} = 3075 m/s$ needs to be calculated. The difference between the two values is $\Delta v_2 = 1457$ m/s.\\

Now all the three delta-Vs are added to the delta-V requirement of phase 1:
\begin{equation}
	\Delta v_{phase_1} = \Delta v_1+\Delta v_1+\Delta v_{inc1} = 5348 m/s
\end{equation}


\subsection{Phase 2 }
\begin{center}
	Transferring the captured satellite to a deorbiting trajectory and returning to LEO at $55^\circ$ inclination
\end{center}
First concept : \\


After capture the satellite needs to be deorbited. Herefore it needs to be set to a drop trajectory. But
since the spacecraft shall not deorbit but stay in LEO it needs to make short correction burn to slightly increase the targeted perigee.
\begin{itemize}
	\item Burn 2.1 : GEO to GTO with (with DROP altitude as perigee, afterwards detaching the satellite)
	\item Burn 2.2 : Inclination change to $55^\circ$
	\item Burn 2.3 : GTO(DROP) to GTO(LEO as perigee)
	\item Burn 2.4 : GTO(LEO) to LEO
\end{itemize}
These $\Delta v$ are calculated in the same way as described above. The total $\Delta v$ for phase 2 with $v_{drop_a} = 1587$ m/s, $v_{aero_{a1}} = 1592$ m/s, $v_{aero_{p1}} = 10283$ m/s, $v_{lto_p}
= 7887$ m/s and $v_{lto_a} = 7596$ m/s.

\begin{align}
	\Delta v_3 &=\Delta v(v_{GEO}, v_{drop_a}) = 1488 m/s\\
	\Delta v_4 &=\Delta v(v_{drop_a}, v_{aero_{a1}}) = 5 m/s   \\
	\Delta v_5 &=\Delta v(v_{aero_{p1}}, v_{lto_p}) = 2396 m/s   \\
	\Delta v_{inc_2} &=\Delta v_i(i, v_{drop_a}) = 1465 m/s   \\
	\Delta v_6 &=\Delta v (v_{lto_a}, v_{leo}) = 72 m/s  \\
	\Delta v_{Phase_2} &= \Delta v_3 + \Delta v_4 + \Delta v_{inc_2} + \Delta v_6 =3031 m/s
\end{align}
Since this maneuver is very costly in terms of fuel consumption the concept of aerobraking was
introduced. An aerobrake is using the atmospheric drag in the upper atmosphere to brake and reduce
velocity. For this mission an aerobrake can save a large amount of delta-V and fuel respectively. in that case, $\Delta v_5$ does not need to be included in the sum, saving 2396 m/s.

\section{Delta-V reduction}
There are further means which enable the reduction of delta-V. The ones applied will be discussed in the
following.\\

\textbf{\underline{Combining burns}}\\
It can make sense to combine burns. Since Inclination changes require burns perpendicular to the flight
direction they can be combined with accelerating or decelerating burns in flight direction. The resulting vector is of shorter length than the sum of the separate burns. Thus, less fuel needs to be burnt. To combine inclination and velocity changing burns the following function was used:
\begin{minted}[fontsize=\footnotesize, linenos, autogobble, breaklines]{matlab}
function a = cosl(v1, v2, i) % combined maneuver of velocity change v1 -> v2 and inclination i change
a = sqrt(v1^2+v2^2-2*v1*v2*cos(deg2rad(i)));
end
\end{minted}

The function uses the Cosine-law to calculate the magnitude of the vector between the original velocity
vector and the target vector rotated by the required inclination.\\

\textbf{\underline{Splitting inclination change burns}}\\
Mostly it makes most sense to make the inclination change at the lowest possible velocity, since the
velocity vector needs to be rotated and this is easier with a shorter vector. However, when the inclination change is split up to combine with further necessary burns the fuel consumption be reduced even further. For the GREDER spacecraft the burns were optimized to reduce the $\Delta V$ to the lowest feasible. The outcome of various optimization loops was to split the inclination change into $1.7^\circ$, $52.6^\circ$ and $1.7^\circ$ as shown in \autoref{tab:missionoption} below.\\

\textbf{\underline{Overshooting the apogee for inclination change}}\\
Since the inclination change is most efficient at lowest velocities it can make sense to overshoot the
targeted apogee and adding a burn to reach the targeted orbit with the inclination already changed. The
inclination change is then performed combined with the added burn to get to the originally targeted orbit.\\

Compare options 3 and 5 in the \autoref{tab:missionoption} below. The higher the orbit the lower is the required $\Delta v$.
However, since increasing the apogee by too much could result in reaching the sphere of influence of the
moon and also increase the mission duration by more than tolerable. Also, the increments by which the
delta-V decreases get smaller by a constant increase of apogee. So as a compromise roughly double the
altitude of GEO was chosen at an orbit radius of 90000 km.\\

\textbf{\underline{Inclination change during aerobrake}}\\
Atmospheric drag can help in reducing the velocity of a spacecraft. In addition, it can be used to generate lift and if directed in the right direction this can reduce or even replace a necessary inclination change burn. An aerobrake strongly influences the architecture of a spacecraft though, since some kind of heat shield is mostly necessary. This is discussed in Chapter 6. Also adding wings to the spacecraft increase its mass and structural behaviour.\\

To estimate the best mission profile several different combinations of these strategies were calculated
and compared to each other. The following \autoref{tab:missionoption} shows the considered missions, the number of burns as well as the total necessary delta-V. To reach the best possible delta-V all the different optimization concepts of the team members were taken into account and combined to settle for a final mission profile.


\begin{table}
	\begin{tabular}{|c|c|c|c|c|}
		\hline
		Option & Phase 1 & Burns & $\Delta v$ (km/s) & Remarks\\
		\hline
		1 & LEO - GTO - INC55 - GEO & 3 & 5.3483 & \\
		\hline
		2 & LEO - GTO - INC55/GEO & 2 & 4.9203 & \\
		\hline
		3 & LEO - INC3/GTO - INC52/GEO & 2 & 4.8796 & \\
		\hline
		4 & \makecell{LEO - LEOto90k -\\ INC55 - 90ktoGEO - GEO} & 4 & 4.9241 & \\
		\hline
		5 & \makecell{LEO - LEOto90k -\\ INC55/90ktoGEO - GEO} & 3 & 4.6916 & \\
		\hline
		6 & \makecell{LEO - INC1.7/LEOto90k\\ - INC51.6/90ktoGEO - INC1.7/GEO} & 3 & 4.6674 & \\
		\hline
		7 & \makecell{LEO - LEOto90k - INC55\\ - 90ktoAERO - AEROtoGEO - GEO} & 5 & 5.0133 & \\
		\hline
		8 & \makecell{LEO - LEOto90k - INC55/90ktoAERO\\ - AEROtoGEO - GEO}  & 4 & 4.9935 & \\
		\hline
		& Phase 2 & & & \\
		\hline
		9 & \makecell{GEO - GEOtoDROP - INC55\\ - GEOtoAERO - AEROtoLEO} & 4 & 3.0309 & \\
		\hline
		10 & \makecell{GEO - GEOtoDROP\\ - GEOtoAERO - AEROtoLEO} & 3 & 1.5655 & \makecell{\small inclination change\\ \small through aerobrake}\\
		\hline
	\end{tabular}
\caption{Mission planning options}\label{tab:missionoption}
\end{table}

Changing the inclination at the perigee with the highest velocity requires a lot more energy than at GEO
altitude. However, if this is achieved without burning fuel and with a lifting body instead a considerable amount of fuel and $\Delta V$ is saved. This resulted in the team decision to settle for the combination of option 6 and 10. The complete mission $\Delta v$ is composed by the following parts:\\
\textbf{\underline{Phase 1}} :
\begin{itemize}
	\item $\Delta v = 4.67$ km/s
\end{itemize}
{\underline{Phase 1 savings}} :
\begin{itemize}
	\item Combined burns : $\approx 430$ m/s
	\item Overshooting the apogee : $\approx 230$ m/s
\end{itemize}
\begin{figure}[H]
	\centering\includegraphics[width=0.6\linewidth]{mission1}
	\caption{Mission planning - First part}
\end{figure}
\textbf{\underline{Phase 2}} :
\begin{itemize}
	\item $\Delta v = 1.57$ km/s
\end{itemize}
{\underline{Further secondary burns}} :
\begin{itemize}
	\item Steering, rotation : $\approx 80$ m/s
	\item Reserve, correction : $\approx 200$ m/s
	\item Capture : $\approx 100$ m/s
\end{itemize}
\underline{\textbf{Total $\Delta v$ for the whole mission}} : $6.61$ km/s
{\underline{Phase 2 savings}} :
\begin{itemize}
	\item Aerobrake : $\approx 390$ m/s
	\item Inclination change during aerobrake : $1470$ m/s
\end{itemize}
\underline{\textbf{Total savings in both phases}} : $2.95$ km/s\\

\autoref{figtim2} shows the paths of the second part of the mission. Here only one exemplary aerobrake orbit is shown as a simplification.
\begin{figure}[H]
	\centering\includegraphics[width=0.7\linewidth]{mission2}
	\caption{Mission planning - Second part}\label{figtim2}
\end{figure}
\chapter{Satellite capturing process}
\qquad \underline{By} : Julien \\
\label{sec:5}
\section{Choice of the process}

\qquad In order to capture and de-orbit a satellite in GEO, we considered the following usable tools :
\begin{itemize}
	\item Net
	\item Harpoon
	\item Claw
	\item Magnet
\end{itemize}

Other solutions such as towing the satellite or simply removing them from GEO were not considered as they either did not fit our program or would create too much strain on our spacecraft.\\

We then took a closer look at the feasibility of each solution and compared the advantages and disadvantages :
\begin{center}
	\begin{tabular}[H]{|c|c|c|c|}
		\hline
		\textbf{Solution} & \textbf{Advantages} & \textbf{Drawbacks} & \textbf{Feasibility}\\
		\hline
		\textbf{Net} & Cheap, simple, low mass &Slow, hard to handle & Yes (JAXA, ESA)\\
		\hline
		\textbf{Harpoon} & Fairly cheap & Can create more debris& Yes (ESA)\\
		\hline
		\textbf{Claw} & Safer & Mechanical, moving parts& WIP (CleanSpace One)\\
		\hline
		\textbf{Magnet} &Adjustable, no moving parts & Higher mass, needs power& Research State\\
		\hline
	\end{tabular}
\end{center}

As the main focus of our mission is reliability and re-usability, we made the choice of using magnets to capture and hold the satellite we would like to de-orbit.

\section{Magnetic solution}
\qquad Even though we decided that we would use magnets, we needed to make sure it was feasible and to lower the drawbacks related to this solution as much as possible. The first precision we need to make is that we will be using electromagnets in order to regulate the intensity of the current in the coil of it, thus, modulating the attraction force so the contact between our spacecraft and the satellite will not be made at high velocity, avoiding damages and space debris creation. \\

Even though it is still at the state of research, we believe that using electromagnets as our capturing solution is realistic as both ESA (with ISAE SupAero) and the NASA have been considering and studying this solution since 2017.\\

However, as a matter of complexity, we will have to make assumptions in order to simplify the problem. The objective in this part is to prove that, with assumptions, this solution can be applied to our mission and to find the required energy to both capture and hold the satellite until its release.
\subsection{Assumptions}
In our calculations, we assumed that :
\begin{enumerate}
	\item We can consider the magnetic circuit between GREDER and the nozzle of the satellite is a closed one (no air gaps)
	\item About $0.1$\% of the satellite's mass is magnetically operable
	\item Mutual attraction is relatively low compared to the magnetic force
	\item Residuals in the alloy of the magnets' cores are neglectable
\end{enumerate}
\section{Sketching and Calculations}
Using the first assumption, we can use the formula for the magnetic force in a magnetic circuit with no air gap :
\begin{equation}
F = \frac{(\mu NI)^2 A}{2\mu_0 L^2}
\end{equation}
\clearpage
With : 
\begin{itemize}
	\item $\mu$ the magnetic permeability of the core of our magnet (determined by the alloy)
	\item $N$ the number of turns of the coil around the core
	\item $I$ the current running through the coil
	\item $A$ the cross section area of the core
	\item $\mu_0$ the magnetic constant
	\item $L$ the length of the mean magnetic circuit
\end{itemize}
In order to have a good balance between thermal properties and magnetic properties we decided to use an alloy made of $90$\% of iron and $10$\% of cobalt for our core. We decided to use this alloy as iron has the best magnetic properties (high relative magnetic permeability) and added cobalt as it has a higher Curie Temperature than iron but has a lower relative magnetic permeability.

In terms of magnet design, we chose to use a squared cross section of $5cm\times 5cm$ for the magnet core and with a length of $15cm$, made of an iron-cobalt alloy and a copper coil around it. We decided to have $135$ turns of the coil around the core with a coil diameter of $1mm$ in order to not have the wire revolutions stuck to each other. We also want to run a current of $10A$ through the coil.\\

\begin{figure}[H]
	\centering
	\includegraphics[width=\linewidth]{magnetCAD}
	\caption{CAD of a capturing magnet}
\end{figure}


With those design choices we get the following parameters while taking into consideration that there should not be any kind of residuals in the core alloy :

\begin{itemize}
	\item $\mu_0 = 4\pi \times 10^{-7}\ H/m$
	\item $\mu = \bigg(\frac{\mu_{iron}+\mu_{cobalt}}{2}\bigg)\times\mu_0=5.868\times10^{-3}\ H/m$
	\item $N=135$ turns
	\item $I = 10\ A$
	\item $A = 25cm^2=2.5\times10^{-3}\ m^2$
	\item $\rho_{core} = \rho_{iron}\times 0.9 + \rho_{cobalt}\times0.1 = 7.9726\ g/cm^3$
	\item $\rho_{coil} = \rho_{copper} = 8.96\ g/cm^3$
\end{itemize}

We then need to find the length $L$ of the mean magnetic circuit. In order to do so, we decided to start the capturing sequence at $10\ m$ from the satellite and that the target has an exploitable nozzle of $40\ cm$ of diameters which is realistic for spacecrafts in the mass range of $3\ 500\ kg$. We could then sketch the capturing sequence as so (not to scale) :

\begin{figure}[H]
	\centering
	\includegraphics[width=\linewidth]{catching}
	\caption{Capturing sequence (not to scale)}
\end{figure}
We can then have the length $L$ as a function of the distance between the tip of our spacecraft and the nozzle of our target. As a result we can proceed to find the feasibility of our solution with this magnet design by finding the time it would require at this state to attract the target. However, in this case, the current modulation when the target is close has not been modeled due to its complexity.\\

Considering that the force will be on one axis only and $m = 0.001\times m_{target} = 3.5\ kg$ :
\begin{align}
\vec F &= m\vec a\\
\frac{(\mu NI)^2 A}{2\mu_0 L(x)^2} &= m \times \ddot x\\
\frac{(\mu NI)^2 A}{2\mu_0 [0.2 + 0.15 + 0.1323 + 2x(t)]^2} &= m \ddot x(t)\\
\frac{(\mu NI)^2 A}{2\mu_0 [0.4823 + 2x(t)]^2} &= m \ddot x(t)
\end{align}

The capturing time can then be found using $ode45$ on Matlab :
\begin{minted}[fontsize=\footnotesize, linenos, autogobble, breaklines]{matlab}
clearvars; clc;
catchtime = 1;
x0 = 10;
while 1
    [t,x] = ode45(@f3,[0:1:catchtime], [x0; 0; 0; 0]);
        if (x(catchtime ,1) >= 2 * x0)
            break
        else
            catchtime = catchtime +1;
        end
end
\end{minted}
And the function used for the $ode$ solver :
\begin{minted}[fontsize=\footnotesize, linenos, autogobble, breaklines]{matlab}
function  [Xdot] = f3(t, X)
mu = 5.686e-3; mu0 = 4 * pi * 10 ^(-7);
N = 135; I = 10; A = 0.05 ^ 2; m = 3500 ;
x = X(1); y = X(2); vx = X(3); vy = X(4);
Fmag = (mu * N * I) ^2 * A / (2 * mu0 *(2 * norm(X(1:2)) + 0.4823) ^ 2 );
Xdot = [vx; vy;  0.001 * Fmag / m; 0]; 
end
\end{minted}
We then get a capture time of $812$ seconds. As a result, we can determine the energy required to operate the magnets aswell as their masses and volumes. We are considering a holding time for approximately half an hour and we also need to verify that the magnets will be able to hold the target while we are de-orbiting.
\chapter{Heat shield}
\qquad A complete deorbiting mission from LEO to GEO and returning to LEO has a very high delta-V requirement of XXX km/s. To reduce this delta-V the GREDER spacecraft will use atmospheric drag in the upper atmosphere at an altitude of 70-120 km to reduce its relative velocity when returning from GEO to LEO. This kind of maneuver is called an aerobrake and has the potential to save a significant amount of fuel and total spacecraft mass. Aerobrakes are commonly used for reentry vehicles. However, these experience very high thermal loads since the kinetic energy is converted to heat. Thus, a high mass for a heat shield is necessary. When the velocity is reduced in small increments the spacecraft can radiate the heat away in between the aerobrakes. The advantage is that fuel is only required for trimming maneuvers to set the targeted perigee radius and not for the deceleration itself. A high number of aerobrakes can be performed with relatively small heat loads if extended mission time is not of a large concern.\\

Since the GREDER spacecraft is unmanned and does not utilize cryogenic fuels the extended mission time is not as critical. A heat shield needs to be developed for the spacecraft to protect it at the areas with the highest heat loads.\\

For first rough calculations the total velocity reduced by the aerobrakes was determined as xxx km/s. The mass of the spacecraft after the deorbiting was estimated at 3500 kg. This results in a kinetic energy of:

\begin{equation}
	E_{kinetic}
\end{equation}

This kinetic energy was then assumed to be completely converted to thermal energy and distributed evenly over the mass of a heat shield. With the maximum allowable temperature of the chosen material a rough required heat shield mass could be estimated.
$$
\text{formula thermal energy, heat capacity, heat shield mass}
$$
\begin{table}
	
	\caption{Comparison of materials for heat shield}
\end{table}

Applying the specific heat capacity of several different flight proven materials carbon fibre reinforce carbon (CFRC) quickly proved as the material of choice for a low heat shield mass. Several metals were also taken into account since the manufacturing and integration into the spacecraft struture would be easier. But the low mass for a CFRC heat shield in combnation with the low thermall conductivity made this the material of choice. CFRC was utilised in the space shuttle heat shield and was found to be the cause of the columbia catastrophe since it is quite brittle and thus sensitive to impact []. But since the GREDER spacecraft will be launched inside a rocket launcher fairing no high velocity impacts are to be expected. The capturing of the satellite on the nose cone will be done at very low relative velocities since the force applied by the magnets is controllable with the applied current and can be carefully adjusted.
\begin{equation}
	m_{shield_{cc}}= \frac{m_{sc} \times (\Delta v_5\times 1000)^2}{2n_{ab}c_{cc}(T_{max_{cc}}-T_0)}
\end{equation}

With :
\begin{itemize}
	\item $\Delta v_5$ : Required $\Delta v$
	\item $m_{sc}$ : Spacecraft mass
	\item $n_{ab}$ : Number of aerobrakes
	\item $T_{max}$ : Maximum allowable temperature of CFRC
	\item $T_0$ : Temperature at beginning of aerobrakes
\end{itemize}
An aerobrake can also be used to change the inclination of the orbital plane. To achieve this the spacecraft needs to be a lifting body. To increase the lift wings can be attached to increase the projeted area of the spacecraft. This area affects both the lift and drag of the spacecraft.
\begin{align}
	F_l &= C_l\times\frac \rho 2 \times v_1^2 \times A_h\\
	\Rightarrow F_l &= 13.9\ kN\\
	F_d &= C_d\times\frac \rho 2 \times v_1^2 \times A_h\\
	\Rightarrow F_d &= 3.5\ kN
\end{align}
The angle of attack influences the lift and drag coefficients. Since both forces are perpendicular to each other the resulting force and the angle of the direction are calculated by:

\begin{align}
	F_{res} &= \sqrt{F_l^2+F_d^2}\\
	F_{res} &= 14.3\ kN\\
	\beta_{F_{res}} &=\arctan\bigg(\frac{F_l}{F_d}\bigg)\times\frac{180}\pi\\
	\beta_{F_{res}} &=76^\circ
\end{align}
It may be noted that unlike an aircraft the lift vector does not point "upwards" but rather parallel to ground and perpendicular to the orbital plane. Thus, the altitude is not affected by the "lift".\\

The lift and drag coefficient are typically experimental values. For the drag similar shapes can be used as a reference. For the GREDER spacecraft a value of $0.3$ is assumed to be reasonable since its similarity to the bullet shape. 
\begin{figure}[H]
	\centering\includegraphics[width=0.7\linewidth]{shapedrag}
	\caption{Shape effect on drag}
\end{figure}
Regarding the lift coefficient a strong simplification as made. For a flat plate the lift coefficient can be approximated with the following formula based on the angle of attack :
\begin{equation}
	C_L = 2\pi\sin(\alpha)
\end{equation}
For $11^\circ$ the value here is $1.2$.\\

To reduce fuel consumption for trimming burns for attitude control a constant angle of attack was chosen. The required delta-Vs for decelerating and inclination change are also perpendicular to each other and the angle between them is given by:
\begin{equation}
	\beta_{res}=\arctan\bigg(\frac{\Delta v_{inc_2}}{\Delta v_5}\bigg) = 75.9^\circ\\
\end{equation}
For an angle of attack of $11^\circ$ the angle of the resulting vector of lift and drag is $76^\circ$. This results in a balance of both forces during each aerobrake.\\

To estimate the altitude of the aerobrakes, the Matlab function atmoscoesa was used. Here the density of the atmosphere is calculated by altitude between $0$ and roughly $84000$ m. The values for higher altitudes are extrapolated. A higher altitude results in a longer necessary time in the atmosphere. This has the benefit of a slower heating up of the spacecraft and low values for lift and drag forces. However this could require a higher number of aerobrakes. Weighing these parameters against each other an altitude of $82$ km was chosen for the aerobrakes. At this altitude the total aerobraking time is $30.4$ minutes. At a number of $20$ aerobrakes each would take roughly 91 seconds. During the first aerobrake the distance traveled would be roughly $937$ km and during the last roughly $730$ km. The forces acted upon the spacecraft are $13.9$ kN for lift and $3.5$ kN for drag. At a remaining scpacecraft mass of $3000$ kg this adds up to an acceleration of $4.8$ km/s$^2$ so roughly $0.5$ Gs. These values seemed reasonable since this would keep the mechanical stress relatively low allowing for reusability of the GREDER spacecraft.\\

Regarding the thermal loads on the spacecraft the heat shield needs a little more detail. Modeling the heat flux and conductance within the heat shield is simplified due to complexity. Since the even distribution of the heat over the shield is inaccurate several conservative assumptions shall leave a margin for this. CFRC retains its mechanical properties up to $2400^\circ$ C. The maximum allowable temperature used for the heat shield modeling is 2000 Kelvin. Also, the assumed temperature at the beginning of the aerorakes is $400$ Kelvin. Furthermore It is assumed that the entire kinetic energy is converted to thermal energy and this is completely distributed onto the spacecraft. In reality however not only the spacecraft but also the atmosphere would be heated through friction. This is highly dependent on the geometry of the body. Most reentry vehicles use a rather flat front face to "push" a protective heat cushion in front of them. This keeps the highest heat loads away from the surface. As a further simplification, heat transfer is considered solely through convection during the aerobrake maneuver and solely through radiation during the completion of each orbit. The short duration of the aerobrakes is not very significant compared to the duration of a completed orbit ($90$ s vs. roughly $90$ minutes).\\

To avoid mechanical failure of the spacecraft hull beneath the heat shield a aerogel layer is introduced due to its excellent isolating properties and very low density. Here a 10 mm layer of silica aerogel blankets at a density of roughly $20$0 kg/m$^3$ is utilized. \\

The thermal loads are expected to be highest on the leading edges of the spacecrafts wings and the nose of the spacecraft. In these areas the heat shield shall have the largest thickness. \textbf{Fig x} shows the areas where the heat shield will be applied. roughly half of the GREDER spacecraft will be covered by the heat shield. This includes the lower faces and leading faces. The upper side and rear of the spacecraft is not expected to endure high thermal loads during the aerobraking maneuvers since the nose will be tilted upwards exposing the lower side.\\

Final mass :
\begin{align}
	m_{shield_cc} &= \frac{m_{sc} \times (\Delta v_5\times 1000)^2}{40\times c_{cc}(T_{max_{cc}}-T_0)} = 344.9\ kg\\
	m_{aerogel} &= A_v\times 0.01\times 200 = 17.3kg
\end{align}
\chapter{Propellant selection}
\qquad \underline{By} : Alina\\

For the propellant selection for our vehicle, several important aspects have to be taken into account including:
\begin{itemize}
	\itemsep0em
	\item 	Specific impulse in vacuum (Isp) of the propellant
	\item	Storability
	\item	Toxicity including ground handling
	\item	Costs (while the main cost driver is not the propellant itself more its toxicity)
	\item	Reasonable refueling options due to the desired mission profile of the vehicle
	\item	Space flight heritage of the propellant (e.g. flight proven, ground proven or in development)
	\item	Density specific impulse
	\item	Corrosive behavior and compatibility with typical light weight tank materials as titanium or aluminum
	\item	and many more
\end{itemize}

At first several possible and typical flight proven propellant combinations were analyzed and the most promising combinations were selected and then compared in detail. Afterwards a feasibility study for the chosen propellant combination was deemed necessary in order to check whether the desired propellant combination is able to perform the required mission and de-risk the next development steps.
\clearpage
\section{Options overview}
\qquad In literature and historical research several reasonable and flight proven propellant combinations have been found. The propellant combinations are divided in three sub-categories: petroleum, cryogenics and hypergolic. These combinations are all bipropellant combinations, the monopropellants have already been excluded at the beginning due considerably low specific impulse in comparison to bipropellants and therefore not suitable for the intended mission profile.\\

Petroleum fuels are containing a combination of complex hydrocarbons and have been refined from mineral oil. The typical petroleum used as rocket fuel is a highly refined kerosene, called RP-1 (rocket propellant 1). In bipropellant use an oxidizer is necessary and therefore petroleum is often mixed with LOX (liquid oxygen). The specific impulse of a petroleum fuel and cryogenic oxidizer combination is higher than for hypergolic propellant combinations but lower than for fully cryogenic options. \\

The cryogenics are gaseous bipropellants stored at very low temperatures and usually stand out due to their high specific impulse. The most common cryogenics are liquid hydrogen and liquid oxygen which have to be stored at $-253^\circ$C for LH2 and $-183^\circ$C for LOX. Recently, cryogenic combinations with liquid methane (LCH4) as fuel are receiving more attention due to availability methane on mars and therefore it might become attractive for future mars missions.\\

The last group in this option overview are the hypergolics. The hypergolics are bipropellants that ignite spontaneously when in contact. The main advantage of hypergolics is the storability. Hypergolics are liquid at room temperature and therefore no additional heating or cooling is necessary during the mission. Nevertheless the hypergolics provide less specific impulses than cryogenics and they are mostly highly toxic.\\

\autoref{tab1} shows a comparison of different bipropellants divided in the three above mentioned categories. In the initial comparison of this chapters the propellant combinations are compared with regards to their major characteristics: specific impulse, flight evidence and therefore technical risk, storability and their reasonability for our application. \\

It is clearly visible that most propellant combinations have not been found reasonable for our application due to low temperature storage. Our vehicle is designed to perform several missions with several ignitions permission and refueling at an in orbit refueling station – this mission profile is not conformable with constant low temperature storage. \\
\clearpage
In contrast to that the group of hypergolics offers great storability and good specific impulse for the mission profile and a storage of these propellants at an in orbit refueling station is feasible. Since the two options MON/MMH and H2O2/RP-1 are promising, further in depth analysis was deemed necessary. 
\begin{table}[H]

	\centering\includegraphics[width=\linewidth]{propcombination}
	
	\caption{Propellant combinations overview}\label{tab1}
\end{table}
\clearpage
\section{Detailed comparison between MON/MMH and H2O2/RP-1}
\qquad The hypergolic propellant combination MON/MMH and H2O2/RP-1 have been found reasonable for our mission and application. \autoref{tab2} shows a detailed comparison of the propellant combinations. Several characteristics were found to be minor and others were found to be major for our application. The major characteristics are the vacuum specific impulse, the tank volume ration, the combined density, the density specific impulse and the toxicity and storability (highlighted in yellow). \\

Both propellant combinations have a comparable specific impulse. Therefore no favor for either MON/MMH or H2O2/RP-1.\\

The tank volume ratio is better for MON/MMH because the ratio is nearly one to one, therefore both oxidizer and fuel tank could share the same tank design. This would significantly decrease the development costs since no second tank design and second qualification tank is necessary. Also the manufacturing costs would decreased due to more possibilities of batch production. Especially production costs of forged tank hemispheres are a huge cost driver and these would decrease due to non-recurring costs being distributed on a larger number of hemispheres.  Also the costs for jigs and tools would decrease. This favors MON/MMH.\\

The combined density for H2O2/RP-1 is slightly better than for MON/MMH. Nevertheless the combined density is not a significant value without taking the specific impulse into account. Calculating the density specific impulse, which is basically the specific impulse per mass, H2O2/RP-1 exceeds MON/MMH by 12\%. In this category H2O2/RP-1 is the winner because it can provide more specific impulse per mass.\\

Comparing toxicity of MON/MMH to H2O2/RP-1 it can be stated that MON/MMH is highly toxic and has to be handled with extreme care which increases the ground handling costs of this propellant combination. In contrast to that is the toxicity of H2O2/RP-1. This propellant combination is non-toxic and can be considered as “green” propellant. Nevertheless careful ground-handling is also necessary for this combination because it is prone to reaction with trace elements. H2O2/RP-1 succeeds in this category.\\

The last category is the storability. Since both propellant combinations are storable in liquid/liquid condition both propellants are suitable for the application.\\
\clearpage
Taking all criteria and results into account H2O2/RP-1 is the preferred propellant combination for our application. The main reasons are:
\vspace{-20pt}
\begin{itemize}
	\itemsep0em
	\item	comparable specific impulse to MON/MMH
	\item	better density specific impulse
	\item	non-toxic advantage for maintenance at refuel station
	\item	good on-ground handling
	\item	possibility of R\&D funds by ESA for green debris remover	
\end{itemize}
\vspace{-20pt}
The main reason against MON/MMH is the toxicity.
\vspace{-20pt}
\begin{table}
	\centering\includegraphics[width=\linewidth]{detailedcompprop}
	\caption{Detailed propellant comparison MON/MMH and H2O2/RP-1}\label{tab2}
\end{table}
\section{H2O2/RP-1 feasibility check}
\qquad Since H2O2/RP-1 is not a common propellant combination and not as frequently used space industry as MON/MHH a further feasibility study has to be performed:
\vspace{-15pt}
\begin{itemize}
	\itemsep0em
	\item	to de-risk the next development steps
	\item	to check whether the desired propellant combination is able to perform the required mission
\end{itemize}
\vspace{-15pt}
In \ref{tab3} historical data of flight H2O2/RP-1 engines is analyzed. One of the engines “Gamma-2” was flown, the others were in development. The thrusts of all engines are comparable or higher than foreseen for our application and two engines also operate in comparable delta-v ranges. This leads to the conclusion that a H2O2/RP-1 engine is generally speaking feasible.

\begin{table}[H]
	\centering\includegraphics[width=\linewidth]{histflightprop}
	\caption{Historical data of flight H2O2/RP-1 engines}\label{tab3}
\end{table}
\section{Summary}

\qquad In summary H2O2/RP-1 is a viable non-toxic hypergolic propellant combination with a good density specific impulse. Additionally the propellant combination is supported by historical flight data. Therefore all further steps will be based on this combination.
\chapter{Vehicle architecture}
\qquad \underline{By} : Lukas\\

This chapter aims to describe the vehicle system architecture. As the space craft needs to be able to perform an aerobreak within earth’s atmosphere, the design is relatively constrained towards a space shuttle-related architecture, with a heat shield in the front, some form of lifting devices and a length to diameter ratio greater than 1. As the satellite capturing system GREDER uses needs to be situated in the front of the vehicle and the propellant choices dictated the performance, the tank volumes were defined by mission requirements and the mass budget. After defining the system diameter to be $2.5$ m, the complete system design was basically finalized. The complete 3D-render can be seen in \autoref{3drender}.

\begin{figure}[H]
	\centering\includegraphics[width=\linewidth]{3drender}
	\caption{Vehicle design 3D-Render}\label{3drender}
\end{figure}

The vehicle consists of 3 zones, which are highlighted in the 3D view as the frontal black zone, the middle white zone and the engine, which is colored in black. The frontal zone houses the battery, fuel cell and capturing equipment as well as the heat shield. A detailed CAD of this area has not been created, as all the equipment was only calculated based on its function and performance. The engine zone, which is the main component of this assignment, will be described in detail, including CAD, calculations and simulations, in \autoref{chap:10} and \autoref{chap:sim}. \\

The tank zone, which consists of the oxidizer tank in the upper part of the cylindric vehicle section and the fuel tank in the lower end, is shown in \autoref{vehtank}.

\begin{figure}[H]
	\centering\includegraphics[width=0.8\linewidth]{vehtank}
	\caption{Vehicle architecture - Tank section}\label{vehtank}
\end{figure}

As \autoref{vehtank} shows, the tanks and the outer wall are the same structure throughout the cylindrical sections of the propellant tanks. Both tanks are shapes as cylinders will elliptical bulkheads, between which the oxidizer tank outlet is situated. The refueling ports are located between the tanks as well. The oxidizer as well as liner tank lines are funneled through the wing, which extends down until the engine section, where the engine is structurally supported by a conically shaped engine frame, which is connected to the cylindrical structure by axial dampers.
\chapter{Mass model}
\section{Mass Budget - First Iteration}

\qquad Before actually going into our mass budget, we wanted to get a reference
idea for the propellant mass so that we would be sure to be able to
achieve our \(\Delta v\). In order to get this, we decided to find a
relation between the usable propellant mass and the mass of the rest as
a ratio. This is then fixed and will also allow us to know roughly how
much propellant we need depending on the dry mass. Let \(m_{UP}\) be the
mass of usable propellant. Moreover, we would be aiming for a total initial mass of roughly $20$ to $25t$ on our last iteration. This first iteration was done with another magnet design, presented in November which consisted in two large discs of $600\ kg$ each and have then been abandoned for the second iteration.


\subsection{Coefficients \& Masses after steps}

Considering that \(ISP = 295s\) and annotating
\(\frac{m_{UP_i}}{m_{total_i}} = K_i\) with \(i\) the burn number :

\begin{longtable}[]{@{}cccc@{}}
\toprule
Step & Required \(\Delta v\) in \(m/s\) & \(K_i\) & Mass after
step\tabularnewline
\midrule
\endhead
1 & 2802.4 & 0.620 & 0.38 \(m_{initial}\)\tabularnewline
2 & 1342.2 & 0.371 & 0.239\(m_{initial}\)\tabularnewline
3 & 522.9 & 0.165 & 0.200\(m_{initial}\)\tabularnewline
Satellite caught & NA & NA & 0.2\(m_{initial}\) + 3500\tabularnewline
4 & 1487.8 & 0.402 & 0.1196\(m_{initial}\) + 2093\tabularnewline
Satellite release & NA & NA & 0.1196\(m_{initial}\) -
1407\tabularnewline
5 & 5.3 & 0.002 & 0.1194\(m_{initial}\) - 1404.186\tabularnewline
6 & 72.4 & 0.0247 &\tabularnewline
\bottomrule
\end{longtable}


\subsection{Global equation between $m_{UP}$ and
	$m_{initial}$}

\begin{longtable}[]{@{}ccc@{}}
\toprule
Step & \(\frac{m_{UP}}{m_{initial}}\) & Bias due to
debris\tabularnewline
\midrule
\endhead
1 & 0.620 & 0\tabularnewline
2 & 0.141 & 0\tabularnewline
3 & 0.039 & 0\tabularnewline
4 & 0.0804 & 1407\tabularnewline
5 & 0.00024 & -2.814\tabularnewline
6 & 0.00295 & -36.68\tabularnewline
TOTAL & 0.88359 & +1369.506\tabularnewline
\bottomrule
\end{longtable}

We then get our general relation between the usable propellant mass and
the initial mass

\[m_{prop} = 0.88359 m_{init} + 1369.506\]

And as \(m_{initial} = m_{UP} + m_{rest}\) :

\[m_{prop} = \frac 1{0.11641}\bigg[0.88359 m_{rest} + 1369.506\bigg]\]

\(m_{rest}\) includes the dry mass and the propellant required for the
ACS.

\subsection{First iteration of mass budget}


\subsubsection{Sub systems}

\begin{longtable}[]{@{}cc@{}}
\toprule
Contributor & Mass in kg\tabularnewline
\midrule
\endhead
\underline{\textbf{EPS}} & -\tabularnewline
Fuel cells & 165.6727\tabularnewline
H2 for fuel cell (tank included) & 10\tabularnewline
Cables & 20\tabularnewline
GNC & 5\tabularnewline
Batteries & 61.3333\tabularnewline
Actuators (for flaps) & 10\tabularnewline
Servos & 1\tabularnewline
\underline{\textbf{On board computer}} & 5\tabularnewline
\underline{\textbf{Telecommunications}} & 10\tabularnewline
\underline{\textbf{Thermal control}} & 10\tabularnewline
\underline{\textbf{ACS/RCS}} & -\tabularnewline
Reaction wheels & 106\tabularnewline
ACS (without propellant) & 36.16\tabularnewline
\underline{\textbf{\emph{Total}}} & 440.166\tabularnewline
\bottomrule
\end{longtable}


\subsubsection{Payload}

\begin{longtable}[]{@{}cc@{}}
\toprule
Contributor & Mass in kg\tabularnewline
\midrule
\endhead
Magnet & 1200\tabularnewline
\bottomrule
\end{longtable}


\subsubsection{Structure}

\begin{longtable}[]{@{}cc@{}}
\toprule
Contributor & Mass in kg\tabularnewline
\midrule
\endhead
Hull & 509\tabularnewline
Wing & 54\tabularnewline
Engine & 60\tabularnewline
Engine frame & 51\tabularnewline
Connectors & 25\tabularnewline
Tanks & 350\tabularnewline
Heat shield & 472\tabularnewline
\underline{\textbf{\emph{Total}}} & 1521\tabularnewline
\bottomrule
\end{longtable}


\subsubsection{Others}

\begin{longtable}[]{@{}cc@{}}
\toprule
Contributor & Mass in kg\tabularnewline
\midrule
\endhead
Catalyzer & 10\tabularnewline
Lines & 25\tabularnewline
ACS including Propellant & 672\tabularnewline
Non usable propellant (Residuals, transient, etc.) & 200\tabularnewline
Helium (including tank) & 30\tabularnewline
\underline{\textbf{\emph{Total}}} & 937\tabularnewline
\bottomrule
\end{longtable}

We then get

\[m_{rest} = m_{Sub systems} + m_{Payload} + m_{Structure} + m_{Others} = 4098.166kg\]

Which, with the previously obtained equation :

\[m_{UP} = 42\ 870.926kg\\\]

As the mixture ratio is $MR = 7.07$ and $m_{UP} = m_{UF} + m_{UOP}$\\
\begin{align*}
m_{UsableFuel} &= \frac{m_{UP}}{1 + MR} = 5\ 312kg\\
m_{UsableOxidizer} &= MR \times m_{UsableFuel} =37\ 559kg
\end{align*}

\subsubsection{Results}
We can sum this first iteration up with the following table :

	\begin{table}[H]
\centering
	
\begin{tabular}[H]{|c|c|}
	\hline
	\cellcolor{gray!50}\textbf{Contributor} & \cellcolor{green!20}\textbf{Mass} (kg)\\
	\hline
	\textbf{Structure} & $1\ 521$\\
	\hline
	\textbf{Magnets} & $1\ 200$\\
	\hline
	\textbf{Sub Systems} & $440.166$\\
	\hline
	\textbf{Tank Pressurization} & $30$\\
	\hline
	\textbf{Engine} & $60$\\
	\hline
	\textbf{Catalyzer} & $10$\\
	\hline
	\textbf{Lines} & $25$\\
	\hline
	\cellcolor{gray!50}\textbf{Dry mass} & \cellcolor{green!20} $3\ 286.166$\\
	\hline
	\textbf{Non usable propellant} & $200$\\
	\hline
	\textbf{ACS/RCS Propellant} & $142. 12$\\
	\hline
	\textbf{Usable propellant} & $42\ 870.926$\\
	\cellcolor{red!50}\textbf{Total initial mass} & \cellcolor{red!50}$46\ 969.092$\\
	\hline 
\end{tabular}
\caption{Initial mass budget}
\end{table}

This first initial mass is way over what we are targeting and there are many parameters to be refined during the next iteration.
\newpage
\section{Mass Budget - Second iteration}
\qquad After refining multiple parameters and fixing others to get more accurate values, we went into the second iteration of our mass budget. Having our $I_{SP}$ changed also required another iteration in our calculation formula between the usable propellant mass the the rest of the mass.

\subsection{Coefficients \& Masses after steps}

Considering that \(ISP = 315s\) and annotating
\(\frac{m_{UP_i}}{m_{total_i}} = K_i\) with \(i\) the burn number :

\begin{longtable}[]{@{}cccc@{}}
\toprule
Step & Required \(\Delta v\) in \(m/s\) & \(K_i\) & Mass after
step\tabularnewline
\midrule
\endhead
1 & 2802.4 & 0.596 & 0.404 \(m_{initial}\)\tabularnewline
2 & 1342.2 & 0.352 & 0.261792\(m_{initial}\)\tabularnewline
3 & 522.9 & 0.156 & 0.221\(m_{initial}\)\tabularnewline
Satellite caught & NA & NA & 0.221\(m_{initial}\) + 3500\tabularnewline
4 & 1487.8 & 0.382 & 0.137\(m_{initial}\) + 2163\tabularnewline
Satellite release & NA & NA & 0.137\(m_{initial}\) -1337\tabularnewline
5 & 5.3 & 0.0017 & 0.1368\(m_{initial}\) - 1334.73\tabularnewline
6 & 72.4 & 0.023 &\tabularnewline
\bottomrule
\end{longtable}


\subsection{\texorpdfstring{Global equation between \(m_{UP}\) and
		\(m_{initial}\)}{Global equation between m\_\{UP\} and m\_\{initial\}}}
\begin{longtable}[]{@{}ccc@{}}
\toprule
Step & \(\frac{m_{UP}}{m_{initial}}\) & Bias due to
debris\tabularnewline
\midrule
\endhead
1 & 0.596 & 0\tabularnewline
2 & 0.142 & 0\tabularnewline
3 & 0.041 & 0\tabularnewline
4 & 0.084 & 1337\tabularnewline
5 & 0.0002 & -2.273\tabularnewline
6 & 0.0032 & -30.699\tabularnewline
TOTAL & 0.8664 & +1304.028\tabularnewline
\bottomrule
\end{longtable}



This time our equation between those two masses is given by 
\begin{equation}
m_{UsableProp} = \frac 1{0.1336}\bigg[0.8664 m_{rest} + 1304.028\bigg]
\end{equation}

\subsection{Second iteration of mass budget}

\qquad As our way of presenting our first iteration of the mass budget didn't seems clear enough to us, we decided to present it in another, more logical way :
\textbf{\underline{Structure}}
\begin{center}
\begin{tabular}[H]{|c|c|}
	\hline
	\cellcolor{gray!50}Contributor & \cellcolor{gray!50}Mass (kg)\\
	\hline
	Hull & $192$\\
	\hline
	Tanks (including non usable propellant) & $700$\\
	\hline
	Wings & $136$\\
	\hline
	Lines & $60$\\
	\hline
	Connectors & $16$\\
	\hline
	$H_2$ tank & $12$\\
	\hline
	\cellcolor{green!30}\textbf{Structure} & \textbf{$1\ 116$}\\
	\hline
\end{tabular}
\end{center}

\textbf{\underline{Electrical related contributors}}
\begin{center}
\begin{tabular}[H]{|c|c|}
	\hline
	\cellcolor{gray!50}Contributor & \cellcolor{gray!50}Mass (kg)\\
	\hline
	Batteries & $241$\\
	\hline
	Fuel cells & $202$\\
	\hline
	On Board Computer & $5$\\
	\hline
	Cables & $20$\\
	\hline
	$H_2$ for fuel cells & $5$\\
	\hline
	Wing actuators & $10$\\
	\hline
	Telecommunications & $10$\\
	\hline
	GNC & $5$\\
	\hline
	Thermal Control & $10$\\
	\hline
	Magnets (Payload) & $25.65$\\
	\hline
	\cellcolor{green!30}\textbf{Electrical related contributors} & \textbf{$533.65$}\\
	\hline
\end{tabular}
\end{center}

\textbf{\underline{ACS and RCS}}
\begin{center}
\begin{tabular}[H]{|c|c|}
	\hline
	\cellcolor{gray!50}Contributor & \cellcolor{gray!50}Mass (kg)\\
	\hline
	Thrusters & $36$\\
	\hline
	$H_2O_2$ & $90$\\
	\hline
	Reaction wheels & $106$\\
	\hline
	\cellcolor{green!30}\textbf{ACS \& RCS} & \textbf{$232$}\\
	\hline
\end{tabular}
\end{center}

\textbf{\underline{Propulsion}}
\begin{center}
\begin{tabular}[H]{|c|c|}
	\hline
	\cellcolor{gray!50}Contributor & \cellcolor{gray!50}Mass (kg)\\
	\hline
	Engine & $93$\\
	\hline
	Turbopumps & $25$\\
	\hline
	Pressurization ($He$) & $1.3$\\
	\hline
	Catalyzer & $30$\\
	\hline
	\cellcolor{green!30}\textbf{Propulsion} & \textbf{$149.3$}\\
	\hline
\end{tabular}
\end{center}

With those tables, we can deduce $m_{rest}$ :
\begin{align}
m_{rest} &= m_{Structure} + m_{Elec} + m_{ACS\&RCS} + m_{Propulsion}\\
m_{rest} &= 2\ 030.95kg
\end{align}
Thus,
\begin{align}
m_{UsableProp} &= \frac 1{0.1336}\bigg[0.8664 m_{rest} + 1304.028\bigg]\\
m_{UsableProp} &= 22\ 931kg\\
m_{Fuel} &= \frac{m_{UsableProp}}{MR+1}\\
m_{Fuel} &= 2841.6kg\\
m_{Ox} &= m_{UsableProp} - m_{Fuel}\\
m_{Ox} &= 20\ 089.89kg\\
m_0 &= 24\ 962.41kg
\end{align}



In this second iteration with a better \(I_{sp}\) and refined values for all of the contributors, we have a large improvement as our initial mass decreased drastically.
\newpage
\section{Frozen information}[H]
After our second iteration of the mass budget, we decided to make a list of the fixed values that we will work around in our further design.
\subsection{Frozen points}
\begin{itemize}
\item We will do $20$ aerobrakes
\item We will have a separate tank design
\item $H_2O_2$ will be pressurized by its decomposition
\item The decomposition control will be managed by rotation of the spacecraft
\item ACS/RCS Layout similar to the Space Shuttle
\item $H_2O_2$ catalyzers separate
\item $H_2O_2/O_2$ separation via thermodynamic properties
\item $H_2/O_2$ will be used in fuel cells to produce energy
\end{itemize}
\begin{table}
\centering
\begin{tabular}[H]{|c|c|c|}
	\hline
	\cellcolor{gray!50}Data & \cellcolor{gray!50}Value & \cellcolor{gray!50}Unit\\
	\hline
	Empty raw mass & $2\ 031$ & kg\\
	\hline
	Usable propellant & $22\ 931$ &kg\\
	\hline
	\cellcolor{green!50}Total mass & \cellcolor{green!50}$24\ 962$ & \cellcolor{green!50}kg\\
	\hline
	Flowrate & $10$ & kg/s\\
	\hline
	Rocket diameter & $2$ & $m$\\
	\hline
	$I_{sp_{vacuum}}$ & $335$ & s\\
	\hline
	Thrust $F=\dot m I_{sp} g_0$ & $32 863.5$ &N\\
	\hline
	Mixture Ratio & $7.07$ & -\\
	\hline
	Wall thickness & $TBA$ & m\\
	\hline
	$H_2O_2$ internal pressure & $1.35$ & bar\\
	\hline
\end{tabular}
\caption{Frozen information}
\end{table}

\newpage
\section{Mass Budget - Final iteration}
As the fixed $I_{sp}$ has been refined as well as other parameters, we went into our final iteration of the mass budget with the same process as the two previous ones.
\subsection{Coefficients \& Masses after steps}

Considering that \(ISP = 335s\) and annotating
\(\frac{m_{UP_i}}{m_{total_i}} = K_i\) with \(i\) the burn number :

\begin{longtable}[]{@{}cccc@{}}
\toprule
Step & Required \(\Delta v\) in \(m/s\) & \(K_i\) & Mass after
step\tabularnewline
\midrule
\endhead
1 & 2802.4 & 0.574 & 0.426 \(m_{initial}\)\tabularnewline
2 & 1342.2 & 0.335 & 0.283\(m_{initial}\)\tabularnewline
3 & 522.9 & 0.148 & 0.241\(m_{initial}\)\tabularnewline
Satellite caught & NA & NA & 0.221\(m_{initial}\) + 3500\tabularnewline
4 & 1487.8 & 0.364 & 0.153\(m_{initial}\) + 2226\tabularnewline
Satellite release & NA & NA & 0.153\(m_{initial}\) -1274\tabularnewline
5 & 5.3 & 0.0016 & 0.1528\(m_{initial}\) - 1271.96\tabularnewline
6 & 72.4 & 0.0218 &\tabularnewline
\bottomrule
\end{longtable}


\subsection{\texorpdfstring{Global equation between \(m_{UP}\) and
		\(m_{initial}\)}{Global equation between m\_\{UP\} and m\_\{initial\}}}



\begin{longtable}[]{@{}ccc@{}}
\toprule
Step & \(\frac{m_{UP}}{m_{initial}}\) & Bias due to
debris\tabularnewline
\midrule
\endhead
1 & 0.574 & 0\tabularnewline
2 & 0.143 & 0\tabularnewline
3 & 0.042 & 0\tabularnewline
4 & 0.088 & 1274\tabularnewline
5 & 0.0002 & -2.038\tabularnewline
6 & 0.0033 & -27.73\tabularnewline
TOTAL & 0.8505 & +1244.232\tabularnewline
\bottomrule
\end{longtable}
Thus,
$$
m_{UsablePropellant} = \frac 1{0.1495}[0.8505m_{rest}+1244.232]
$$
\subsection{Detailed contributors}
\subsubsection{Structure}
\begin{center}
\begin{tabular}[H]{|c|c|}
	\hline
	\cellcolor{gray!50}Contributor & \cellcolor{gray!50}Mass (kg)\\
	\hline
	Hull & $192$\\
	\hline
	Tanks & $700$\\
	\hline
	Wings & $136$\\
	\hline
	Lines & $70$\\
	\hline
	Connectors & $16$\\
	\hline
	Brackets & $35$\\
	\hline
	$H_2$ Tanks & $12$\\
	\hline
	\cellcolor{green!30}\textbf{Structure} & \textbf{$1161$}\\
	\hline
\end{tabular}
\end{center}
\subsubsection{Electrical systems}
\begin{center}
\begin{tabular}[H]{|c|c|}
	\hline
	\cellcolor{gray!50}Contributor & \cellcolor{gray!50}Mass (kg)\\
	\hline
	Batteries & $388.66$\\
	\hline
	Fuel cell & $202$\\
	\hline
	OBC & $10$\\
	\hline
	Cables & $57$\\
	\hline
	$H_2$ for fuel cells & $5$\\
	\hline
	Wing actuators & $10$\\
	\hline
	Data transmission & $10$\\
	\hline
	GNC & $10$\\
	\hline
	Thermal control & $20$\\
	\hline
	Magnets & $25.65$\\
	\hline
	\cellcolor{green!30}\textbf{Electrical systems} & \textbf{$738.31$}\\
	\hline
\end{tabular}
\end{center}
\subsubsection{Attitude control}
\begin{center}
\begin{tabular}[H]{|c|c|}
	\hline
	\cellcolor{gray!50}Contributor & \cellcolor{gray!50}Mass (kg)\\
	\hline
	Thrusters & $36$\\
	\hline
	$H_2O_2$ for the ACS & $90$\\
	\hline
	
	Reaction wheels & $106$\\
	\hline
	\cellcolor{green!30}\textbf{Attitude control} & \textbf{$232$}\\
	\hline
\end{tabular}
\end{center}
\subsubsection{Propulsion}
\begin{center}
\begin{tabular}[H]{|c|c|}
	\hline
	\cellcolor{gray!50}Contributor & \cellcolor{gray!50}Mass (kg)\\
	\hline
	Engine & $93$\\
	\hline
	Turbopumps + Electric motors & $170$\\
	\hline
	$He$ for pressurization & $1.3$\\
	\hline
	Catalyzer & $40.86$\\
	\hline
	\cellcolor{green!30}\textbf{Propulsion} & \textbf{$305.16$}\\
	\hline
\end{tabular}
\end{center}
\subsection{Final mass budget}

\begin{table}[H]
	\centering
\begin{tabular}[H]{|c|c|}
	\hline
	\cellcolor{gray!50}Contributor & \cellcolor{gray!50}Mass (kg)\\
	\hline
	Structure & $1161$\\
	\hline
	Electrical systems & $738.31$\\
	\hline
	Attitude control & $232$\\
	\hline
	Heat shield & $360$\\
	\hline
	Propulsion & $305.16$\\
	\hline
	\cellcolor{green!30}\textbf{$m_{rest}$} & \textbf{$2796.5$}\\
	\hline
	\cellcolor{green!30}\textbf{$m_{UP}$} with a $5\%$ performance window & \textbf{$25\ 505$}\\
	\hline
	\cellcolor{green!30}\textbf{$m_{Fuel}$}  & \textbf{$3160. 5$}\\
	\hline
	\cellcolor{green!30}\textbf{$m_{Ox}$}  & \textbf{$22\ 345$}\\
	\hline
	\cellcolor{red!30}\textbf{Initial wet mass}  & \textbf{$28\ 302$}\\
	\hline
\end{tabular}
\caption{Final mass budget}
\end{table}

\chapter{Propulsion system}
\label{chap:10}
\section{Engine cycle}
\qquad \underline{By} : Lukas\\

The spacecraft uses electrically-driven turbo pumps to feed the oxidizer as well as the fuel to the engine. The propellant and oxidizer are each driven out of their tanks at low pressure, where after a turbo pump in each propellant line strain raises their pressures. As the oxidizer strain has a higher mass flow rate and faces a large pressure drop in the catalyzer, the respective pump is also more powerful as a result. The turbo pumps are driven electrically by electric motors which use large batteries for their power intake. These batteries offer enough charge for one maximum burn time of 900 seconds, after which they are re-powered by a fuel cell which runs on hydrogen and the hydrogen peroxide decomposition product oxygen. This will be explained further in \autoref{sec:10-3}. In order to demonstrate how exactly the engine cycle is made up, a flow schematic is shown in the following. Firstly, the pressurization system is shown in \autoref{fig1}.
\begin{figure}[H]
	\centering\includegraphics[width=0.5\linewidth]{pressurizationsystem}
	\caption{Flow Schematic - Pressurization system}\label{fig1}
\end{figure}

As the figure shows, only the RP-1 tank is pressurized by pressurization gas, using a $300$ bars helium tank. The hydrogen peroxide has certain decomposition characteristics which enable it to self-pressurize due to the rising pressure upon vaporization. The critical point of hydrogen peroxide is at around $150$ degrees Celsius and $1.5$ bars, meaning that if the thermal control is sufficiently reliable, a tank pressure of around $1.3$ bars can be maintained by self-pressurization. The control system and more details will be explained in \autoref{sec:10-3}. The $300$ bars H2 Tank that can be seen in the pressurization system flow schematic is therefore not a pressurization tank, but a tank for the sole purpose of running the fuel cell in combination with the oxygen which is separated from water, which is the second decomposition product of hydrogen peroxide. The separation works by simply condensing the water and allowing the gaseous oxygen to pass through a filter. Both the RP-1 tank and the H2O2 tank have a main valve after their outlets, which are visible in \autoref{fig1}. The remaining feeding system is shown in the second part of the flow schematic, depicted in \autoref{fig2}.

\begin{figure}[H]
	\centering\includegraphics[width=0.7\linewidth]{flowenginesection}
	\caption{Flow Schematic - Engine section}\label{fig2}
\end{figure}
The fueling ports for RP-1 as well as hydrogen peroxide extend to the left- and right-hand-side of the top of the figure. Check valves are situated at these points to only allow propellant flowing in. A second main valve for both propellants is installed just before the turbopumps, which are closed during refueling. While the RP-1 is then funneled through cooling channels in the regenerative cooling system, the oxidizer runs into a catalyzer, where a rapid decomposition reaction splits the hydrogen peroxide into its reactive products for combustion. A second oxidizer strain is guided towards a pressure regulation valve, behind which it continues into a buffer tank of hydrogen peroxide for monopropellant use in the ACS. The catalyzers for ACS thrusters are located in close proximity to their respective combustion chambers. The ACS is not depicted as a flow schematic.

\section{RCS / ACS}
\qquad \underline{By} : Alexis\\

In the previous section we mention that we are going to use hydrogen peroxide as a monopropellant for our Reaction Control System. Indeed, hydrogen peroxide can be used as a quite good RCS propellant. \\

At the moment, the biggest part of the monopropellant thruster using $H_2O_2$ are test bench engine. This is because of the difficulty to characterize the engine and its parameters. \\

Hydrogen peroxide is usable as a monopropellant because of its natural decomposition. As we are going to see on the Catalyzer part further on the report $H_2O_2$ can be decomposed in $H_2$ and $H_2O$. That's this decomposition we are going to use inside our thruster. \\
When hydrogen peroxide goes through a catalyser it decomposes and generates a great amount of heat, up to $1000K$. This two factors create steam at a very high temperature and pressure; it's this steam that creates our thrust.\\

Before designing the engine and its characteristics we need to choose the positioning of the thrusters. In order to do that, we based our design on the American Space Shuttle which use cluster of small thrusters all around the spacecraft to allow a good maneuverability.  

\begin{figure}[H]
    \centering
    \includegraphics[width=\linewidth]{shiprcs}
    \caption{RCS thruster repartition}
\end{figure}

We are going to use 6 different clusters spread over the craft, each cluster is composed of 2 thrusters for a total of 12. This disposition allow to manipulate every axis. \\

Now we need to compute the characteristic of the thruster, to do so we used RPA (a NASA software to compute the parameters of a thruster based on the propellant and several other characteristics) and some papers of recent studies about hydrogen peroxide thruster. \\

We assumed a chamber pressure of $10bars$ and a thrust of $100N$, then we use RPA to get the other parameters.

\begin{itemize}
    \item Combustion temperature: $1223K$
    \vspace{-0.4cm}
    \item Ejection temperature: $424K$
    \vspace{-0.4cm}
    \item Ejection pressure : $0.094bars$
    \vspace{-0.4cm}
    \item $\gamma = 1.335$
    \vspace{-0.4cm}
    \item $R = 368.6$
    \vspace{-0.4cm}
    \item $ISP = 140s$
\end{itemize}

With these parameters we can then compute every other parameters we want for the thruster and especially the mass flow rate which is necessary to have a correct mass budget. 

$$c^* = \sqrt{\frac{R T_c}{\gamma}}\times \frac{\gamma + 1}{2}^{\frac{\gamma + 1}{2(\gamma - 1)}}$$

Throat characteristics:
$$T_t = T_c \left(\frac{2}{\gamma + 1}\right) \hspace{20pt}
P_t = P_c \left(\frac{2}{\gamma + 1}\right)^{\frac{\gamma}{\gamma - 1}} \hspace{20pt}
\rho = \frac{P_t}{R T_t} \hspace{20pt}   u_t = \sqrt{\gamma R T_t}$$

Exhaust characteristics:
$$M_e = \sqrt{\frac{2}{\gamma - 1}\left[\left(\frac{P_c}{P_e}\right)^\frac{\gamma - 1 }{\gamma} - 1 \right]} \hspace{20pt} T_e = \frac{T_c}{1 + \frac{\gamma - 1}{2}M_e^2}  \hspace{20pt} \rho_e = \frac{P_e}{R T_e}$$

\clearpage

Thus we can compute the area ratio and the thrust coefficient:

$$\frac{A_e}{A_t}= \frac{1}{M_e}\left[\frac{2}{\gamma + 1}\left(1 + \frac{\gamma - 1}{2} M_e^2 \right)\right]^{\frac{\gamma + 1}{2(\gamma - 1)}}$$

$$ C_F = \gamma \sqrt{\left(\frac{2}{\gamma + 1}\right)^{\frac{\gamma + 1}{\gamma - 1}}\frac{2}{\gamma - 1} \left[ 1 - \left(\frac{P_e}{P_c}\right)^{\frac{\gamma - 1}{\gamma}}\right]} + \frac{P_e - P_a}{P_c}\frac{A_e}{A_t} $$ 

\vspace{0.5cm}

Finally we can compute the exhaust velocity, throat area and the mass flow: 

$$c = C_F c^* \hspace{20pt} A_t = \frac{F}{C_F P_c} \hspace{20pt} \dot m = \frac{P_c A_t}{c^*}$$

Thus we can write the following Matlab code to simulate:

\begin{minted}[linenos, autogobble, breaklines]{matlab}
Pc = 10; % Chamber pressure
Pe = 0.094; % exhaust Presssure (given RPA)
gamma = 1.335; % RPA
R = 0.3686e3; % RPA
Tc = 1223; % Combustion temperature given (RPA)

c = sqrt( R * Tc / gamma) * ((gamma + 1)/2)^((gamma + 1)/2 * (gamma - 1));

% Throat characterisctics
Tt = Tc * (2/(gamma + 1));
Pt = Pc * (2/(gamma + 1))^(gamma/(gamma -1));
rho = Pt / R * Tt;
Ut = sqrt(gamma * R * Tt);

% Exhaust characteristics
Me = sqrt((2/(gamma - 1))*((Pc/Pe)^((gamma - 1)/ gamma) -1));
Tebis = Tc / (1 + ((gamma - 1) / 2) * Me^2);
rho_e = Pe / R * Tebis;

Exp_ratio = (1 / Me) * ((2 / (gamma + 1)) * (1 + ((gamma - 1) / 2) * Me^2))^((gamma + 1)/2*(gamma-1));
Cf = gamma * sqrt((2/(gamma + 1))^((gamma + 1)/(gamma - 1)) * (2/(gamma - 1)) * (1 - (Pe/Pc)^((gamma - 1)/gamma))) + ((Pe - 1)/Pc) * Exp_ratio;

c_e = Cf * c; At = 100 / (Cf * Pc); m_flow = (Pc * At)/c;
\end{minted}

From the simulation we finally have:

\begin{itemize}
    \item Exhaust velocity: $c=617.38m/s$
    \vspace{-0.4cm}
    \item Mass flow rate: $\dot m = 0.105 kg/s$
\end{itemize}

The most important value for us is the mass flow rate as it allows us to compute the propellant mas needed to perform our maneuvers. \\


In addition of the thrusters we also need another system which is the Reaction Wheels. This systems which is separated in 3 different wheel, one for each axis, and allow to rotate the entire craft on the 3 axis with the the reaction motion applied by the rotation of the wheel.

\section{Multi-usage of hydrogen peroxide}
\qquad \underline{By} : Lukas\\

\label{sec:10-3}
Hydrogen Peroxide as the main engines oxidizer is the main dictator of all sub-system and system design decisions. GREDER uses it for following applications:
\begin{enumerate}
	\item	Main engine oxidizer
	\item	ACS mono-propellant
	\item	Fuel cell power generation in combination with stored hydrogen
	\item	Oxidizer tank pressurization
\end{enumerate}

Hydrogen peroxide’s decomposition behavior is the reason for its multi-purpose usability, but also its major risk. If the pressure or temperature in the tank go unregulated and exceed certain boundaries, the propellant can enter critical status and endanger the tank’s integrity. Therefore, extensive design and simulation was conducted to determine a tank design which allows thermal regulation based on the rotational angle of space craft, as indicated in \autoref{fig:tempreg}. 

\begin{figure}[H]
	\centering
	\includegraphics[width=\linewidth]{tempreg}
	\caption{Lessons learnt}\label{fig:tempreg}
\end{figure}

The absorbing side of the space craft is painted in a way which causes a high absorption coefficient and a low emissivity, while the radiative side is coated vice-versa. The tank materials are chosen to deliver high heat transfer while being compatible with hydrogen peroxide. Thereby, a rotation of the space craft will cause a change in the heat flux, while every hydrogen peroxide temperature can be assigned a rotational angle for neutral heat flux at which no temperature change happens. This design was simulated with some simplifications and its technological feasibility could be verified. This is detailed in \autoref{sec:11-3}. \\

The usage of hydrogen peroxide is the sole reason for GREDER using electrically-driven turbo pumps, as the ease in electricity production makes this the most efficient method. The relatively low combustion chamber pressure and thrust are not a major issue, as a long mission duration is necessary anyways for electricity production in between burns. In addition, the low complexity of the electric turbo pump cycle compensates for the higher complexity of the challenge of storing hydrogen peroxide for long durations, as it is a highly reactive chemical which poses extreme risks, especially due to its decomposition behavior, which GREDER uses as an advantage. \\

As batteries have a low energy density compared to chemical fuels, electric cycles for main engines usually come with a mass disadvantage due to the necessity of carrying large battery masses. While GREDER needs a total main engine burn time of around 2500 seconds per mission, the battery only needs to power the turbo pumps and electrical equipment for 900 seconds at a time, because the hydrogen peroxide decomposition product oxygen powers a fuel cell in combination with hydrogen between main engine burns for additional power generation. The main trade-off to consider at this point is between bringing a fuel cell, a hydrogen tank and the additional equipment for separation of the decomposition products and dimensioning the battery large enough for powering the entire mission. As the other applications of the hydrogen peroxide are discussed in other chapters, this chapter will detail the power generation usage and detail a trade-off with respect to using a larger battery. The detailed description of how the power generation and tank pressure control work can be found in chapter 12 and the simulation model in the annex. This chapter will use analytical terms to compare the two approaches, while the simulation can be seen as a proof of concept.\\

In order to determine the necessary masses for the additional components for fuel cell usage, the energy output of the fuel cell was calculated using the reaction energy multiplied with an assumed efficiency coefficient supported by literature. Following assumption for the efficiency was made:

\begin{equation}
	\eta_{Fuel_{cell}} = 0.6
\end{equation}
With the theoretical reaction energy being :
\begin{equation}
	P_{Fuel\ cell_{th}} = 286\ 000 \ J/mole
\end{equation}
For the reaction :
\begin{center}
	$2H_2\ +\ O_2\ =\ 2H_2O$
\end{center}

Taking the mole masses into account and assuming complete reaction, for every kilogram of hydrogen the fuel cell uses $7.937$ kg of oxygen. The resulting energy per kilogram of hydrogen can then be calculated as follows:
\begin{equation}
	E_{Fuel\ cell_{real}} = 286\ 000\ J/mole \times 0.6\times\frac{1kg}{0.01\ kg/mole} = 47.29\ kWh
\end{equation}
In order to perform a trade-off calculation between using a fuel cell for battery re-charging or installing a larger battery directly, the total masses for both systems will be compared at this point. The current mass budget for the fuel cell system is as follows:
\begin{table}[H]
	\centering
	\begin{tabular}{|c|c|}
		\hline
		Component & Mass (kg)\\
		\hline
		Battery & $388.66$\\
		\hline
		Fuel cell & $202$\\
		\hline
		$H_2$ tank & $5$\\
		\hline
		Other equipement & $10$\\
		\hline
		\cellcolor{gray!50}Total & \cellcolor{gray!50}$605.66$
	\end{tabular}
\end{table}

Now, as the fuel cell allows GREDER to only be designed for a $900$ s duration of not being re-powered, the battery mass can remain relatively low at a conservative battery energy density of $100$  Wh/kg. During the burn duration, around $98.6$\% of electrical energy is used to power the turbo pumps. Therefore, as a first estimate, the necessary battery mass for powering the turbo pumps during all mission burn times, which amount to $2475.2$ seconds, is calculated:
\begin{equation}
	m_{Bat,\ only bat} = \frac{P_{TP}\times 2475.2s}{100\ Wh/kg} = 1\ 053.9\ kg
\end{equation}

This results in a \textbf{mass difference of 448.2kg in favour of the fuel-cell architecture}. Now, two possible problems can be brought up with this estimation:
\begin{enumerate}
	\item	The energy density estimate is at the lower end of current lithium battery technology and a higher density would result in a shift towards the advantages of using only a battery.
	\item	The fuel-cell architecture is not technologically proven in combination with hydrogen peroxide decomposition and constitutes larger technical complexity.
\end{enumerate}

While these points are valid arguments for using only a battery, the trade-off analysis clearly pointed out using a fuel-cell would result in a mass reduction, the higher level of technical complexity of which is justifiable. The above-mentioned arguments can be answered as follows:
\begin{enumerate}
	\item	While lithium batteries of higher energy densities have been flight-proven, the mission itself has never been performed in a similar way. Refuelling and frequent satellite capturing are completely new territory and therefore, conservative estimates in key design criteria offer a way of compensating for the additional mission challenges.
	\item	Fuel cells are very well researched and very common technology in on-earth use. Therefore, the implementation of a decomposition-product recycling system, while posing some new design problems, is not comparable to the difficulty of challenges like designing a new engine. Additionally, as this mass estimate only takes turbo pump power into account, the battery-only architecture would need to either have even greater battery mass or be fitted with solar power generators to cover the system power requirements between burns. This again adds up to system complexity and poses large problems during re-entry.
	
\end{enumerate}
\section{Propellant tanks}
\qquad \underline{By} : Alina\\

The vehicle must use propellant tanks in order to carry the oxidizer and fuel, in this case H2O2 and RP-1, during the mission. The required total useable propellant was already found in the final mass budget calculation. 
This chapter describes the tank design beginning with the requirements and tank specification, afterwards the analysis of compatible materials following by the design including different options, expulsion principle, stresses, MAIT and final design. The Liner tanks are also addressed in the last part of this chapter.\\

\subsection{Requirements and specification}
Several parameters have to be set in prior to the propellant tank design in order to ensure that all top level vehicle and propulsion system requirements are met. Additionally it must be ensured that the propellant tank assemblies sustain launch and flight loads during all phases of the mission.
\subsubsection{Propellant Tank masses and volumes}

\autoref{tab:propmass} shows the calculated values of the final propellant mass and final volumes of both oxidizer and fuel. The basis of this calculation is the final mass budget that provided the useable propellant masses. Starting at this value and calculating 1\% additional propellant for ignition, 2\% for shutdown, 2\% performance reserve and 2\% residuals, the masses add up to 24t for H2O2 and 3.4t for RP-1. H2O2 needs a greater ullage volume than RP-1 due to the decomposition, with 20\% ullage a total tank volume of $\geq$ 18.62 m$^3$ is necessary. For RP-1 a total tank volume of  $\geq3.99$ m$^3$ needs to be accomplished.
\begin{table}[H]
    \centering
    \includegraphics[width = \linewidth]{propmassvol}
    \caption{Propellant masses and volumes}
    \label{tab:propmass}
\end{table}{}
\subsubsection{Driving requirements}
The driving requirements for the oxidizer propellant tank assembly (PTA) are summarized in \autoref{tab:PTAreq}. MDP is taken according to ECSS\footnote{ECSS‐E‐ST‐32‐01C Rev. 1, Fracture control}
\begin{table}[H]
    \centering
    \includegraphics[width = \linewidth]{ptareq}
    \caption{PTA driving requirements}
    \label{tab:PTAreq}
\end{table}{}\pagebreak
\subsubsection{Pressure loads}
The ECSS\footnote{ECSS‐E‐ST‐32‐01C Rev. 1, Fracture control} requires a safety factor of 1.25 on proof pressure and 1.5 on burst pressure. Therefore the pressure levels including safety factors are listed in \autoref{tab:ptapress}.
\begin{table}[H]
	\centering
	\includegraphics[width = \linewidth]{ptapress}
	\caption{PTA pressure loads}
	\label{tab:ptapress}
\end{table}{}
The number of pressure cycles after delivery is shown in \autoref{tab:ptacycle}. This number of pressure cycles applies for both the oxidizer and fuel tank.
\begin{table}[H]
	\centering
	\includegraphics[width = 0.6\linewidth]{ptapress}
	\caption{PTA pressure cycles}
	\label{tab:ptacycle}
\end{table}{}
\pagebreak
\subsection{Materials}
The intended materials for the propellant tanks are listed in \autoref{tab:tankmat}. For the fuel tank and liner tanks a common titanium alloy can be used with high strength and good ductility. The titanium alloy shall be heat treated to .7 status (solution treated and aged) in order to achieve the below mentioned material characteristics and to stress relieve. The liner tanks can also be winded or wrapped with CFK in order to be further strengthened and to reduce the wall thickness of the titanium.\\

The oxidizer tank shall be manufactured of aluminum 5254 which is long term compatible and corrosion resistant against H2O2. Unfortunately the aluminum alloy shows a low strength which has to be taken into account in the mechanical calculations. A higher strength option of this alloy is currently in development acc. to NASA paper.
\begin{table}[H]
	\centering
	\includegraphics[width = \linewidth]{tankmat}
	\caption{Propellant tank materials}
	\label{tab:tankmat}
\end{table}{}
\pagebreak
\subsection{Concepts and options}
During the process of finding the optimum propellant tank concept for the GREDER vehicle, our mission and application, several possible design option have been taken into account. \autoref{fig:tankdesign} shows the three most promising options.\\

The first option is an integrated concept with the fuel tank placed within the oxidizer tank. Integrated designs are in theory the most lightweight design option for propellant tanks because the basic idea is to reduce the wall thickness of the inner tank due to theoretically no pressure delta between fuel and oxidizer tank pressure. The downside of this option is the manufacturing. Also the use of spacecraft volume is quite attractive with this design. Manufacturing of one tank inside of another tank and then welding it is a tough task. Additionally the fuel tank also needs to be compatible with both propellants because it is in contact with the oxidizer from the outside and the fuel from the inside. Therefore option 1 is not suitable for our application due to manufacturing complexity, material compatibility and our MDP of 1.3 bar is way too low to justify the advantage of wall thickness reduction.\\

The second option is also an integrated option but this time with two separate tanks. This slightly increases the spacecraft volume but is still more spacecraft volume efficient than option 3. This option, especially the special shape of the oxidizer tank, is a complex geometry and therefore also complex to manufacture. The fuel tank manufacturing is simpler than in the first option due to the half sphere hemispheres and gimbal mounting. During first attempts of a detailed design of that option it was found that the vehicle diameter increased over the boundary requirement. For this reason and also due to the complex manufacturing and design this option was found to not be the most efficient design for our application.\\

The third option is a rather simple “two-tanks-on-top-of-each-other” design. The main advantage is here the simple design and manufacturing and assembly of all parts, the possibility of load carrying structure of the outer tank walls as well as good options for tank expulsion principles. This design was found to be the optimum for our application and is therefore the baseline for all further calculations and detailed design.

\begin{figure}[H]
	\centering
	\includegraphics[width=\linewidth]{tankdesignoption}
	\caption{Different propellant tank design options}\label{fig:tankdesign}
\end{figure}
\subsection{Final design}
The final design is therefore an option with a stacked structure, as already mentioned in the previous chapter. \autoref{fig:detail} shows a detailed view on the final design with the major parts shown. The Oxidizer tank consists of two cassini shaped hemispheres and one large cylinder (or several smaller cylinders if easier to manufacture). The fuel tank is built with the same cassini hemispheres, but without a cylinder. This usage of the same hemispheres for both oxidizer and fuel tanks reduces costs for jigs and tools and also for development. Each tank has an inlet at the top of the tank and an outlet at the bottom. It has to be mentioned that the upper port of the oxidizer tank is used to extract the H2O2 decomposition products and not to pressurize the tank since the tank is self-pressurizing.

Both propellant tanks make use of start baskets, the function is explained in the subsequent chapter.\\

The propellant tanks are the carrying structure of the spacecraft and they are connected by interface connector rings. In the interface connector ring there are also the adapters for refueling planned.\\

At the bottom of the spacecraft near the engine are the liners located. There is one helium liner tank for Fuel tank pressurization and one hydrogen liner tank for the fuel cell operation.
\autoref{fig:designfull} shows a full view on the final design of the propellant tanks.
\begin{figure}[H]
	\centering
	\includegraphics[width=\linewidth]{tankdetail}
	\caption{Final propellant tank design detailed view}\label{fig:detauk}
\end{figure}
\begin{figure}[H]
	\centering
	\includegraphics[width=0.4\linewidth]{tankfull}
	\caption{Final propellant tank design full view}\label{fig:designfull}
\end{figure}
\subsection{Expulsion principle}
The refillable start basket is a structure within the propellant tanks located at the outlet of the tank that is designed to ensure that there is always propellant at the outlet in zero gravity condition – at least until the spacecraft acceleration again forces the propellant towards the outlet. If the thrusting duration is sufficient to settle the liquid each time, the start basket, is the best light weight and simple propellant management device. The start basked volume for each of the tanks must be big enough (with margin) to supply propellant to the engine until acceleration settles the liquid again. Another important aspect is that the start basket must be refillable to allow several ignitions of the engine. This principle is shown in \autoref{fig:startbasket}.
\begin{figure}[H]
	\centering
	\includegraphics[width=0.3\linewidth]{startbasket}
	\caption{Final propellant tank design full view}\label{fig:startbasket}
\end{figure}
\subsubsection{Start basket design}
For the start basket design, the maximum acceleration has to be defined. As already listed in the propellant tank requirements, the maximum acceleration is:
\begin{align}
    g_x &= 4\times g_0\\
    g_z &= 2\times g_0\\
    g_0 &= 9.81 m/s
\end{align}{}

The first step for the start basket design is the calculation of settling acceleration, the acceleration with which the propellant moves towards the outlet
\begin{equation}
    a_{set} = \frac{F}{m_{spacecraft}} = \frac{32860N}{28302kg} = 1.16m/s^2
\end{equation}{}
With this acceleration the time the propellant needs to reach the ground can be calculated
\begin{align}
    t_{fall_{Oxidizer}} &= \sqrt{\frac{h_{Ox}}{\frac{1.16}{2}}} = \sqrt{\frac{4.74}{\frac{a_{set}}{2}}} =2.54s\\
    t_{fall_{Fuel}} &= \sqrt{\frac{1.2}{\frac{1.16}{2}}} = 2.54s
\end{align}{}

Since the time that the oxidizer and fuel need to reach the outlet is now known, the settling time can be calculated. The settling time is the falling time multiplied by four. This is a heritage value
\begin{align}
    t_{set_{Oxidizer}} &= 4\times t_{fall_{Oxidizer}} = 10.16s\\
    t_{set_{Fuel}} &= 4\times t_{fall_{Fuel}} = 5.76s
\end{align}
With this information, the necessary start basket volume can be calculated by calculating what propellant volumes are needed to cover the settling time while still providing the requested flow rates to the engine.
\begin{align}
    V_{basket_{Oxidizer}} &= \frac{t_{set_{Oxidizer}} \times \dot{m_{Oxidizer}}}{\rho_{Oxidizer}} = 0.0618m^3\\
    V_{basket_{Fuel}} &= \frac{t_{set_{Fuel}} \times \dot{m_{Fuel}}}{\rho_{Fuel}} = 0.0079m^3
\end{align}{}
It is now possible to roughly calculate size of the start basket by assuming it to be a cylinder with $d=h$
\begin{align}
    d_{Oxidizer} &= h_{Oxidizer} = \bigg(\frac{V_{basket_{Oxidizer}}}{\frac{4}{\pi}}\bigg)^{\frac{1}{3}} = 0.36m\\
    d_{Fuel} &= h_{Fuel} = \bigg(\frac{V_{basket_{Fuel}}}{\frac{4}{\pi}}\bigg)^{\frac{1}{3}} = 0.18m
\end{align}{}

Afterwards the screens and screen drillings have to be dimensioned in such a way that the screens are capable to hold the propellant inside of the start basked by propellant surface tension. The if the pressure on the grids due to hydrostatic pressure by $g_x$ (the highest acceleration) exceeds the bubble point pressure, the liquid will break through and is not captured inside the start basket anymore

\begin{align}
    p_{BubblePoint_{Oxidizer}} = \rho_{Oxidizer}\times g_x \times h_{Oxidizer} = 206mbar\\
    p_{BubblePoint_{Fuel}} = \rho_{Fuel}\times g_x \times h_{Fuel} = 65mbar\\
\end{align}{}
In the last step the screen drillings are dimensioned in that way that the bubble point is not exceeded by the acceleration taking the surface tension into account
\begin{align}
    Stens_{Oxidizer} &= 0.08 N/m\\
    Stens_{Fuel} &= 0.028 N/m\\
    d_{drillings_{Oxidizer}} &= \frac{4\times Stens_{Oxidizer}}{p_{BubblePoint_{Oxidizer}}} = 0.0155mm\\
    d_{drillings_{Fuel}} &= \frac{4\times Stens_{Fuel}}{p_{BubblePoint_{Fuel}}} = 0.0172mm\\
\end{align}{}
Therefore it has to be a very fine mesh. 
\subsection{Stresses}
For the preliminary calculation of stresses and wall thicknesses several loads on the propellant tank have been identified:
\begin{itemize}
    \item   Internal tank pressure
    \item	Hydrostatic pressure on tank walls due to accelerated propellant
    \item	Tension and compression stress due to empty mass and propellant mass
\end{itemize}{}
General rules : 
\begin{enumerate}
    \item No rupture at burst pressure
    \item No yield at proof pressure
\end{enumerate}{}

The results of these calculations are listed in \autoref{tab:tankstress}. The calculations are based on the maximum possible wall thickness that still complies with the tank mass requirement. On basis of the results the wall thickness can be further optimized.
\begin{table}[H]
    \centering
    \includegraphics[width = 0.95\linewidth]{tankstress}
    \caption{Preliminary tank stresses calculation}\label{tab:tankstress}
\end{table}{}
\subsection{Liner tanks}
\subsubsection{Helium liner tank}

Only the fuel tank is pressurized by Helium and only needs a tank pressure of 1.3bar. The H2O2 tank is self-pressurizing due to the decomposing characteristics of H2O2. The specification of the helium liner tank is shown in \autoref{tab:helium}
\begin{table}[H]
    \centering
    \includegraphics[width = \linewidth]{heliumliner}
    \caption{Helium liner tank specification}\label{tab:helium}
\end{table}{}
\subsubsection{$H_2$ liner tank}
The H2 is needed for the fuel cell. The specification of the h2 liner tank is shown in \autoref{tab:h2liner}.
\begin{table}[H]
    \centering
    \includegraphics[width = \linewidth]{h2liner}
    \caption{Helium liner tank specification}\label{tab:h2liner}
\end{table}{}

\section{Catalyzer}
\qquad \underline{By} : Alexis\\

Besides the advantages of $H_2O_2$ there is one major drawback that we have to take into account. This drawback is that $H_2O_2$ needs to be decomposed in $H_2$ and $H_2O$ in order to react with RP-1. 

\begin{figure}[H]
	\centering
	\includegraphics{H2O2}
	\caption{$H_2O_2$ chemical decomposition process}
\end{figure}

This decomposition is natural but at a low rate whereas we need a very high decomposition rate in order to feed the combustion chamber and sustain a proper flame. This decomposition is an exothermic decomposition. That's why we need a catalyser. \\

This catalyser needs to be placed between the turbo pump and the injectors. It allows to decompose the $H_2O_2$ at the last time. \\

The way a catalyser works is pretty simple; the $H_2O_2$ goes through a catalyst bed of silver pellets, reacts and generates heat. 
Why silver ? We chose silver because is the mostly used catalyser for $H_2O_2$. However, a lot of other different material exists, like Platinum, Manganese or even Gold but these materials are rarely used due to their cost and also the fact that they need to be made in complex alloy in order to optimize the reaction. 

\begin{figure}[H]
	\centering
	\includegraphics{catalyst}
	\caption{Example of catalyst bed}
\end{figure}

The figure above shows basically how a catalyser works. To have an idea of the shape of ours we just have to swap the ceramic catalyst by silver catalyst. So, it will be a steel cylinder filled with small spherical silver pellet separated by some baffles (silver grid mesh). \\

An important characteristic of the catalyser is the pressure drop it creates. This pressure drop influences the whole feeding system, the turbo pump sizing and even the injector design. that's why we need to characterize the pressure drop created by the catalyser. In order to do so, we are going to use the Ergun equation for packed bed reactor:

$$
\frac{\Delta p}{L} = 151.2 \frac{\mu}{d^2}\frac{(1-\epsilon)^2}{\epsilon^2}u + 1.8 \frac{\rho}{d}\frac{1- \epsilon}{\epsilon^3}u^2
$$

With $\mu$ the dynamic viscosity, $\epsilon$ the porosity, $d$ the pellet diameter, $\rho$ the density, $L$ the length of the bed and $u$ the velocity. \\

In this equation we need some important component such as $\epsilon$. The porosity is complicated to compute and need to be model. According to a recent research we determined a porosity of $0.3802$ with a pellet diameter of 5mm.

\begin{figure}[H]
	\centering
	\includegraphics[width=\linewidth]{pressuredrop}
	\caption{Pressure drop depending on the size of the catalyst bed}
\end{figure}

We see on the graph that the pressure drop rise quickly with the size of the bed. So we need to limit the size of our catalyser in order to not oversize the turbo pump. To do so, we chose to limit our pressure drop to around 30 bars. Finally we obtain a cylinder of 20 cm of diameter and 20 cm of length. \\

This geometry allows the catalyser to provide a sufficient decomposition rate, a "contained" pressure drop and size. It also generates a great heat as high as $1000K$ at the exit of the catalyst bed. 

\section{Injectors}
\subsection{Choice of injector type}
\qquad \underline{By} : Alexis\\

To design our injectors we made some research and went through the literature about injector for hypergolic injector. We found interesting results in this paper\footnote{The design and main performance of a hydrogen peroxide/kerosene coaxial-swirl injector in a lab-scale rocket engine (2017)}. Their goal was to design a small attitude control system hydrogen peroxide/RP-1 thruster, to do so they compared two main different kind of injectors; the coaxial shear injector and the coaxial-swirl injector. \\

After simulation the results were that coaxial shear causes the combustion chamber to be divided into three different zones, these zones are: rapid high-temperature pyrolysis, oxidization and equilibrium flow. This kind of separation can cause serious bad behaviors during the combustion process and creates issues like flame-out, explosion or very poor efficiency. \\
In order to avoid these kind of behaviors the use of swirl-coaxial injectors is the best choice to make. \\

\subsection{Swirl injector functionnality}
\qquad \underline{By} : Alexis\\

The concept of the swirl injector is pretty simple. Indeed, it's based on a coaxial injection but instead of injecting the two propellant in the axis of the injector, here the two propellant are injected radially in what is called the vortex chamber. 
The radial injection combine with the chamber allows to create a vortex of $H_2O_2$ and RP-1 which result in a generation of a plume of propellant at the exit of the injector. In addition, due to the geometry of such an injector, the reduction of diameter at the exit, the atomization is very easy.

\begin{figure}[H]
    \centering
    \includegraphics[width=\linewidth]{swirl}
    \caption{Cross section of a swirl injector}
\end{figure}

The schematic\footnote{\url{https://www.sciencedirect.com/science/article/abs/pii/S127096381730353X}} before shows how a swirl injector works. We see that there is only few parameters to take into account to design this kind of injector. These different parameters will be explain later on. \\

Our injection will be a liquid-gas injection but this characteristic is not really a problem for a swirl injector as the two propellant will behave the same way as a liquid or a gas.\\

in addition of these advantages the swirl injector offers an other non-negligible advantages that it allow us to improve our efficiency. Indeed, the study show that a swirl injector increases the combustion stability and efficiency up to 10$\%$ which result in a better fuel consumption during the operating time of the spacecraft. 
\pagebreak
\subsection{Injector design calculations}
\qquad \underline{By} : Julien\\

Based on historical data, we assumed types of injectors to calculate the injection velocity at the limit between the combustion chamber ($P_{chamber} = 40\ bars$) and the injector. Then, with 

\begin{itemize}
	\itemsep0em 
	\item $ P = P_{chamber} = 40$ bars
	\item $\rho_{Ox} = 1450$ kg/m$^3$
	\item $\rho_{Fuel} = 810$ kg/m$^3$
	\item $\dot{m} = 10$ kg/s
	\item $\dot{m}_{Ox} = 1.239$ kg/s
	\item $\dot{m}_{Fuel} = 8.761$ kg/s
	\item $C_D = 0.9$ (short tube with conical entrance of $1$ mm)
	\item $k_{Drop} = $ from $0.05$ to $0.15$ depending on the type of injector
\end{itemize}
\begin{align}
\Delta P &= k_{Drop}\times P\\
w_{inj_{F,O}} &= C_D \sqrt{2\frac{\Delta P}{\rho_{F,O}}} \\
\dot{V_{F,O}} &= \frac{\dot{m}_{F, O}}{\rho_{F,O}}\\
A_{F, O} &= \frac{\dot{V_{F,O}}}{w_{inj_{F,O}}}
\end{align}
Thus, while taking the average over multiple examples, we get :
\begin{align}
A_O & = 2.75\times 10^{-4} m^2\\
A_{F} &= 5.21\times 10^{-5} m^2
\end{align}
After this, we can compute our velocities
\begin{align}
w_{inj_{O}} &= \frac{\dot{V_O}}{A_O} = 21.9707 m/s\\
w_{inj_{F}} &= \frac{\dot{V_F}}{A_F} = 29.3632 m/s
\end{align} 
We can also get our pressure drop at injection which will be needed in \autoref{sec:10-7}. We choose a sharp edged orifice for $d>2.5$ mm for the fuel and a short time with conical entrance for the oxidizer.
\begin{align}
C_{D_{Ox}} &= 0.84\\
C_{D_F} &= 0.61\\
\Delta P_{Oxidizer\ injection} &= \bigg(\frac{w_{inj_{Ox}}}{C_{D_{Ox}}}\bigg)^2 \times \frac{\rho_{Ox}}{2} = 4.9599\ bars\\
\Delta P_{Fuel\ injection} &= \bigg(\frac{w_{inj_F}}{C_{D_F}}\bigg)^2 \times \frac{\rho_F}{2} = 9.3843\ bars 
\end{align}

\subsection{Injection system concept}
\qquad \underline{By} : Tim\\

The injection system is based on the mass flow and the pressure drop in the injector. The injection area was calculated previous and is :
\begin{align}
A_O & = 2.75\times 10^{-4} m^2\\
A_{F} &= 5.21\times 10^{-5} m^2
\end{align}

Since a single injector in the middle of the injection head would not distribute the fuel/oxidizer mixture well throughout the diameter of the relatively large combustion chamber it makes sense to distribute several injectors over the area of the injection head. A high number of injectors would result in tiny injectors though, since the diameter of a single injector is relatively small compared to the injection head. To reach a well distribution it was decided to use 7 injectors. One in the center of the injection head and the other 6 circling the central one. This results in a constant distance between all neighboring injectors. \autoref{fig:injhead} shows the arrangement of injectors in the injection head. The section is oriented against the direction of the mass flow.

\begin{figure}[H]
	\centering\includegraphics[width=0.8\linewidth]{injhead}
	\caption{Injector head}\label{fig:injhead}
\end{figure}

The swirl injector used, swirls fuel and oxidizer separately. To ensure the separation of fuel and oxidizer before the chamber both are channeled in separate layers. The swirls are then injected in 4 tangential inlets. The following figures show how the injection head geometry is foreseen. The fuel channels are highlighted in orange and the oxidizer channels in blue.

\begin{figure}[H]
	\centering\includegraphics[width=0.8\linewidth]{sectioncc}
	\caption{Section of the combustion chamber}
\end{figure}
\begin{figure}[H]
	\centering\includegraphics[width=0.5\linewidth]{isoview}
	\caption{Iso view of (transparent) injection head}
\end{figure}
\begin{figure}[H]
	\centering\includegraphics[width=0.5\linewidth]{headsection}
	\caption{Section of injection head}
\end{figure}
\section{Feeding system}
\qquad \underline{By} : Julien\\
\label{sec:10-7}
After having designed most of our propulsion system. We need to carefully link them by designing our feeding system. The biggest challenge is to create a system that will both fit in our spacecraft and deliver the right amount of propellant from the tanks to the engine through our different, required other subsystems.\\

The general pressure loss in a system is given by :

$$
\Delta P = K\frac \rho 2 w^2
$$

With $K$ depending on the type of change in system geometry as follow : 
\begin{figure}[H]
	\centering
	\includegraphics[height=12cm]{pertecharge}
	\caption{$K$ values for geometry changes}
\end{figure}
In our case, we will use the progressive line entering loss $K = 0.04$ and the bends $ K(\alpha) = \sin^2(\alpha) + 2\sin^4\bigg(\frac\alpha 2\bigg)$. Another $K$ will also be used for the entrance of the injector, which will be specified later on.\\

For manufacturing costs and simplicity purposes, we choose to only use $45^\circ$ bends which will result in $K_{bends}=0.5429$.\\

With that and the length measurements in mind, we designed the following feeding system layout for which we will then calculate the pressure variations along it :
\begin{figure}[H]
	\centering
	\includegraphics[height=10cm]{feeding}
	\caption{Feeding system layout (To scale)}
\end{figure}
\begin{figure}[H]
	\centering
	\includegraphics[height=8cm]{feedingzoom}
	\caption{Feeding system layout - Zoomed (To scale)}
\end{figure}
\subsection{Line diameters}
In order to choose our line diameters, we can first use our volume flow and then get the line area from it and then at the end, the line diameter.\\
\underline{Fuel} 
\begin{align}
\dot{V_f}&= \frac{\dot{m_f}}{\rho_f} = 0.0015298m^3/s\\
A_{line_f} &= \frac{\dot V_f}{w_f}=5.21\times 10^{-5} mm^2\\
d_{line_f} &= 2\sqrt{\frac{A_{line_f}}{\pi}} = 8.1447mm
\end{align}
As we are trying to insure a high injection velocity and to insure a certain margin in pressure and velocity, we will choose a line diameter of $7$ mm for the fuel feeding system.
\underline{Oxidizer}
\begin{align}
\dot{V_o}&= \frac{\dot{m_o}}{\rho_o} = 0.006042m^3/s\\
A_{line_o} &= \frac{\dot V_o}{w_o}=0.000275 mm^2\\
d_{line_o} &= 2\sqrt{\frac{A_{line_o}}{\pi}} = 18mm
\end{align}
In this case, we will choose a line diameter of $15$ mm for the oxidizer.

\subsection{Fuel feeding system}
The following items in our fuel feeding system will cause pressure drops :
\begin{itemize}
	\item Tank exit : $K=0.04$
	\item 4 $\times$ $45^\circ$ bends : $K = 0.5429$ for each
	\item Straight line losses : $\Delta P = \frac \rho 2 w^2 f \frac{L}{D}$
	\item Friction coefficient : $f = 0.02$
	\item Regenerative cooling : $\Delta P = 0.25$ bar
	\item Fuel injection : $\Delta P = 9.3843$ bars
\end{itemize}

With our current layout, we have $5$ straight lines which will cause pressure losses on the fuel side. Three of them are before the turbopump which is placed at the end of the third straight section, right before the third bend. We consider a velocity of $8$ m/s before the turbopumps and of $v_{inj}=29.363$ m/s after. The line loss for each section is given by : 
$$
\Delta P = \frac {810} 2 w^2 \times 0.02 \times \frac{L_{section}}{0.007}
$$
\begin{enumerate}
	\item First section ($L_{section}=0.05m$) : $\Delta P = 0.037029$ bar
	\item Second section ($L_{section}=0.1m$) : $\Delta P = 0.074057$ bar
	\item Third section ($L_{section}=0.5m$) : $\Delta P = 0.37029$ bar
	\item Fourth section ($L_{section}=0.1m$) : $\Delta P = 0.997699$ bar
	\item Fifth section ($L_{section}=0.05m$) : $\Delta P = 0.49884$ bar
\end{enumerate}
There are also two different values for the bend losses depending on the position of the bend (before or after the turbopump), we have $2$ of each :
\begin{itemize}
	\item $\Delta P_{before} = 0.14072 $ bar
	\item $\Delta P_{after} = 1.8957$ bar
\end{itemize}
The tank exit loss is :
$$
\Delta P_{exit} = K_{exit} \times \frac{\rho_F}2 \times 8 ^ 2 = 0.010368\text{ bar}
$$
\subsection{Oxidizer feeding system}
The following items in our oxidizer feeding system will cause pressure drops :
\begin{itemize}
	\item Tank exit : $K=0.04$
	\item 4 $\times$ $45^\circ$ bends : $K = 0.5429$ for each
	\item Straight line losses : $\Delta P = \frac \rho 2 w^2 f \frac{L}{D}$
	\item Friction coefficient : $f = 0.02$
	\item Catalyzer : $\Delta P = $ bars
	\item Oxidizer injection : $\Delta P = 4.9599$ bars
\end{itemize}

On this part of the feeding system, we also have $5$ sections and $4$ bends of $45^\circ$ each. We also consider a velocity of $8$ m/s before the turbopump and $21.971$ m/s after. However, due to the larger distances (due to our tank layout), the turbopump's position in the feeding system is different and is now positioned after the first bend, right at the beginning of the second straight line. This results in $3$ bends being at high velocity and $1$ at relatively slower velocity.\\

Here, each straight line loss section is given by : 
$$
\Delta P = \frac {1450} 2 w^2 \times 0.02 \times \frac{L_{section}}{0.015}
$$
With : 

\begin{enumerate}
	\item First section ($L_{section}=0.05m$) : $\Delta P = 0.030933$ bar
	\item Second section ($L_{section}=1.95m$) : $\Delta P = 9.0992$ bars
	\item Third section ($L_{section}=1.4836m$) : $\Delta P = 6.9228$ bars
	\item Fourth section ($L_{section}=1.15m$) : $\Delta P = 5.3662$ bars
	\item Fifth section ($L_{section}=0.6m$) : $\Delta P = 2.7997$ bars
\end{enumerate}

For the bends, we have :

\begin{itemize}
	\item $\Delta P_{before} = 0.2519 $ bar ($1$ of them)
	\item $\Delta P_{after} = 1.0614$ bar ($3$ of them)
\end{itemize}

The tank exit loss is :
$$
\Delta P_{exit} = K_{exit} \times \frac{\rho_o}2 \times 8 ^ 2 = 0.01856\text{ bar}
$$
\section{Turbo pumps}
\qquad \underline{By} : Julien\\

As most of our subsystems have a defined pressure drop due to their specific design, we have made the choice to use this feeding system design with all losses included to then design our turbopumps to have a pressure rise in accordance with our pressure requirements. We chose to go with electrically driven turbo pumps as we have a good amount of electrical power since we use fuel cells in our spacecraft.\\

Our respective turbopump required created pressures are : 
\begin{itemize}
	\item Fuel side : $\Delta P_{T_f} = P_{Chamber} + \Delta P_{feeding_f} + \Delta P_{inj_f} + \Delta P_{Regenerative\ cooling} - P_{Tank_f}$
	\item Oxidizer side : 	$\Delta P_{T_o} = P_{Chamber} + \Delta P_{feeding_o} + \Delta P_{inj_o} + \Delta P_{Catalyzer} - P_{Tank_o}$
\end{itemize}
Thus,
\begin{align}
\Delta P_{T_f} &= 54.395\ \text{bars}\\
\Delta P_{T_o} &= 102.28\ \text{bars}
\end{align}
Considering an efficiency of $0.9\times 0.75$, with $0.9$ for the electrical part and $0.75$ for the mechanical part, we get the following powers :
\begin{align}
	Power_{fuelpump} &=\dot{m_f}\frac{P_{T_f}}{\rho_F \eta} = 13\ 461\ W\\
	Power_{Oxpump} &= \dot{m_o}\frac{P_{T_o}}{\rho_o \eta} = 96\ 030\ W
\end{align}
Considering a maximum continuous burn time of $900$ seconds, we get the following energy with a 40\% margin as we are still unsure about the performance of such turbopump :\\
\begin{equation}
	E_{kWh} = \frac{(Power_{fuelpump} + Power_{Oxpump})\times 900}{3.6\times 10^6} = 38.322\ kWh
\end{equation}

We also know the following vapor pressures :
\begin{itemize}
	\item $H_2O_2$ : $666.612$ Pa at $30^\circ$ C
	\item $RP-1$ : $700$ Pa between $20^\circ$C and $25^\circ$ C
\end{itemize}
With the information we have, we can calculate the pump head rise and the NPSH.
\begin{align}
	H_{p_{Fuel}} &= \frac{\Delta p_{p_{Fuel}}}{g_0 \rho_{Fuel}} = 747.474\ m\\
	H_{p_{Ox}} &= \frac{\Delta p_{p_{Ox}}}{g_0 \rho_{Ox}} = 754.192\ m\\
	NPSH_{Fuel} &= \frac{p_{i_{Fuel}} - p_{v_{Fuel}}}{g_0 \rho_{Fuel}} = 6.569\ m\\
	NPSH_{Ox} &= \frac{p_{i_{Ox}} - p_{v_{Ox}}}{g_0 \rho_{Ox}} = 7.0465\ m
\end{align}
We can then get the number of stages :
\begin{equation}
	n= 1 +floor(\frac{\Delta p_p}{\Delta p_{ps}})
\end{equation}
Thus, using $\Delta p_{ps} = 47\times 10^6\ Pa$,
\begin{align}
	n_{Fuel} &= 126\ 373
	n_{Ox} &= 228\ 256
\end{align}
We can also get the rotation speeds :
\begin{align}
	N_{Fuel} &= 1.636\ rad/s = 15.623\ RPM\\
	N_{Ox} &= 0.532\ rad/s = 5.079\ RPM
\end{align}
Then
\begin{align}
	u_{t_{Fuel}} &= \sqrt{\frac{gH_{p_{Fuel}}}{n\psi}}=0.325\ m/s\\
	u_{t_{Ox}} &= \sqrt{\frac{gH_{p_{Ox}}}{n\psi}} =0.243\ m/s\\
	D_{2t_{Fuel}} &=\frac{u_{t_{Fuel}}}{N_{r_{Fuel}}}= 0.199\ m\\
	D_{2t_{Ox}} &= \frac{u_{t_{Ox}}}{N_{r_{Ox}}} 0.457\ m\\
	D_{1t_{Fuel}} &= \sqrt[3]{\frac{\frac{4}{\pi}Q_{Fuel}}{\phi N_{r_{Fuel}}(1-L^2)}} = 2.197\ m\\
	D_{1t_{Fuel}} &=\sqrt[3]{\frac{\frac{4}{\pi}Q_{Ox}}{\phi N_{r_{Ox}}(1-L^2)}} = 6.131\ m
\end{align}
\section{Pressure evolution summary}
\qquad \underline{By} : Julien\\
\subsection{Fuel side}
\begin{table}[H]
	\centering
\begin{tabular}[H]{|c|c|c|}
	\hline
	\cellcolor{gray!50}Contributor& \cellcolor{gray!50}Pressure Drop (bars) & \cellcolor{gray!50}Pressure at the end of this part (bars)\\
	\hline
	Tank & NA & $1.3$ \\
	\hline
	Tank exit & $0.010368$ & $1.29$\\
	\hline
	First section & $0.037$ &$1.253$\\
	\hline
	First bend &$0.14$ &$1.113$\\
	\hline
	Second section &$0.074$ &$1.039$\\
	\hline
	Second bend &$0.14$ &$0.899$\\
	\hline
	Third section &$0.37$ &$0.529$\\
	\hline
	Turbo pump & $54.395 $ (Rise) &$59.924$\\
	\hline
	Valve & $5$ &$54.924$\\
	\hline
	Third bend &$1.8957$ &$53.0283$\\
	\hline
	Fourth section &$0.997$ &$52.0313$\\
	\hline
	Fourth bend &$1.8957$ &$50.1356$\\
	\hline
	Fifth section &$0.499$ &$49.6366$\\
	\hline
	Cooling &$0.25$ &$49.3866$\\
	\hline
	Injection &$9.38$ &$40.0066$\\
	\hline
	Combustion chamber & NA &$40.0066$\\
	\hline
\end{tabular}
\caption{Pressure evolution on fuel side}
\end{table}
\begin{figure}[H]
	\centering
	\includegraphics[width=\linewidth]{fuelchart}
	\caption{Pressure evolution on fuel side (bars)}
\end{figure}
\subsection{Oxidizer side}
\begin{table}[H]
	\centering
\begin{tabular}[H]{|c|c|c|}
	\hline
	\cellcolor{gray!50}Contributor& \cellcolor{gray!50}Pressure Drop (bars) & \cellcolor{gray!50}Pressure at the end of this part (bars)\\
	\hline
	Tank & NA & $1.3$ \\
	\hline
	Tank exit & $0.010368$ & $1.29$\\
	\hline
	First section & $0.031$ &$1.259$\\
	\hline
	First bend &$0.25$ &$1.009$\\
	\hline
	Turbo pump & $102.28 $ (Rise) &$108.289$\\
	\hline
	Turbo pump & $5$ &$103.289$\\
	\hline
	Second section &$9.09$ &$94.199$\\
	\hline
	Second bend &$1.06$ &$93.139$\\
	\hline
	Third section &$6.9228$ &$86.2162$\\
	\hline
	Third bend &$1.06$ &$85.1562$\\
	\hline
	Fourth section &$5.366$ &$79.7902$\\
	\hline
	Fourth bend &$1.06$ &$78.7302$\\
	\hline
	Fifth section &$2.7997$ &$75.9305$\\
	\hline
	Catalyzer &$30.95$ &$44.9805$\\
	\hline
	Injection &$4.96$ &$40.0205$\\
	\hline
	Combustion chamber & NA &$40.0205$\\
	\hline
\end{tabular}
\caption{Pressure evolution on fuel side}
\end{table}
\begin{figure}[H]
	\centering
	\includegraphics[width=\linewidth]{oxchart}
	\caption{Pressure evolution on oxidizer side (bars)}
\end{figure}
\section{Engine}
\qquad \underline{By} : Tim\\


\section{Nozzle}
\qquad \underline{By} : Tim\\


\chapter{Simulation}
\section{Subsystem simulation}
\section{System simulation}
\section{Simulation preparation and execution}
\section{Simulation review}
\chapter{Evaluation}
\qquad \underline{By} : Alina
\section{Requirement verification}
The top level requirements were already defined at the beginning of the project. In the end of the project it was verified whether the requirements were fully fulfilled, partially fulfilled or not fulfilled (See \autoref{tab:reqverif}). Some of the requirements have to be confirmed (TBC), since the result can only be stated after first launch.
\begin{figure}[H]
	\centering
	\includegraphics[width=0.85\linewidth]{reqveh}
	\caption{Requirements for the vehicle}\label{tab:reqverif}
\end{figure}

\section{Lessons learnt}
At the end of the project, a lessons learned session was performed to summarize the major problems and issues
during the whole project work. The result can be found in \autoref{fig:lesson}.
\begin{figure}[H]
	\centering
	\includegraphics[width=\linewidth]{lessonlearnt}
	\caption{Lessons learnt}\label{fig:lesson}
\end{figure}
\include{Conclusion}
\appendix




%\glsaddall
%\printglossaries

%\nocite{*}

	
%	\bibliographystyle{plain} % Le style est mis entre accolades.
%	\bibliography{references} % mon fichier de base de données s'appelle bibli.bib

%\printbibliography


\begin{acronym}[EHPAD] % Give the longest acronym here

\end{acronym}

\include{lexique_an_fr}
\listoffigures

\listoftables
\nopagebreak
\chapter*{Annexes}
\section*{Annex I - Gantt Schedule}
\label{sec:annex1}
\begin{figure}[H]
	\centering
	\includegraphics[width=\linewidth]{gantt1}
\end{figure}
\pagebreak
\thispagestyle{empty}
\begin{figure}[H]
	\centering
	\includegraphics[width=\linewidth]{gantt2}
	\includegraphics[width=\linewidth]{gantt3}
	\caption{Gantt Schedule}
\end{figure}

\chapter*{Sources}

\begin{enumerate}
	\itemsep0em 
	\footnotesize{
	\item http://braeunig.us/space/propel.htm
	\item http://astronautix.com/l/loxkerosene.html
	\item https://en.wikipedia.org/wiki/Liquid\_rocket\_propellant\#Bipropellants
	\item http://www.astronautix.com/h/h2o2kerosene.html
	\item Hydrogen Peroxide / Kerosene, Liquid-Oxygen / Kerosene, and Liquid-Oxygen / Liquid
	Methane for Upper Stage Propulsion, Paper, S. Krishnan* Universiti Teknologi Malaysia
	\item https://ntrs.nasa.gov/archive/nasa/casi.ntrs.nasa.gov/19720019028.pdf
	\item https://www.nasa.gov/centers/glenn/technology/fuel\_cells.html
	\item https://www.ginerinc.com/regenerative-fuel-cell-systems
	\item https://ntrs.nasa.gov/archive/nasa/casi.ntrs.nasa.gov/19990063763.pdf
	\item http://www.esa.int/esapub/bulletin/bullet90/b90dudle.htm
	\item https://www.sciencedirect.com/science/article/abs/pii/S0360319904003106
	\item http://citeseerx.ist.psu.edu/viewdoc/download?doi=10.1.1.526.3797\&rep=rep1\&type=pdf
	\item http://juser.fz-juelich.de/record/135459/files/HP3d\_3\_Ranjbari.pdf
	\item https://www.sciencedirect.com/topics/chemistry/proton-exchange-membrane-fuel-cells
	\item http://www.dartmouth.edu/~cushman/courses/engs37/FuelCells.pdf
	\item https://ocw.mit.edu/courses/aeronautics-and-astronautics/16-851-satellite-engineering-fall-2003/le
	cture-notes/l3\_scpowersys\_dm\_done2.pdf
	\item Article : The design and main performance of a hydrogen peroxide/kerosene coaxial-swirl injector in a lab-scale rocket engine
	\item Article : Design and Analysis of a Fuel Injector of a Liquid
	Rocket Engine
	\item Article : The spray characteristic of gas-liquid coaxial swirl injector by experiment
	\item https://www.grc.nasa.gov/www/k-12/airplane/shaped.html
	\item http://brennen.caltech.edu/fluidbook/externalflows/lift/flatplateairfoil.pdf
}
\end{enumerate}

\end{document}